\ifx\mainclass\undefined
\documentclass[cn,11pt,chinese,black,simple]{../elegantbook}
% 微分号
\newcommand{\dd}[1]{\mathrm{d}#1}
\newcommand{\pp}[1]{\partial{}#1}

% FT LT ZT
\newcommand{\dtft}[1]{\text{DTFT}[#1]}
\newcommand{\dtftr}[1]{\text{DTFT}^{-1}[#1]}
\newcommand{\dtfta}{\xrightarrow{\text{DTFT}}}

\newcommand{\where}[1]{\Big|_{#1}}
\newcommand{\abs}[1]{\left| #1 \right|}
\newcommand{\zt}[1]{\mathscr{Z}[#1]}
\newcommand{\ztr}[1]{\mathscr{Z}^{-1}[#1]}
\newcommand{\zta}{\xrightarrow{\mathscr{Z}}} 
\newcommand{\lt}[1]{\mathscr{L}[#1]}
\newcommand{\lta}{\xrightarrow{\mathscr{L}}} 
\newcommand{\ft}[1]{\mathscr{F}[#1]}
\newcommand{\fta}{\xrightarrow{\mathscr{F}}} 
\newcommand{\dsum}{\displaystyle\sum}
\newcommand{\aint}{\int_{-\infty}^{+\infty} }

% 积分求和号

% 简易图片插入
\newcommand{\qfig}[3][nolabel]{
  \begin{figure}[!htb]
      \centering
      \includegraphics[width=0.6\textwidth]{#2}
      \caption{#3}
      \label{#1}
  \end{figure}
}

\usepackage{pdfpages}
\includepdfset{pagecommand=\thispagestyle{plain}}
% 表格
\renewcommand\arraystretch{1.5}

\usepackage{shapepar}
\usepackage{longtable}
% 日期

\begin{document}
\fi 
\def\chapname{04dft}

% Start Here
\chapter{离散傅里叶变化}

数字处理需要对时域进行采样与截断,频域离散化。

从离散傅里叶级数(DFS)开始,逐步介绍离散傅里叶变换(DFT),以及快速算法。

\section{离散傅立叶级数}

周期序列等价于有限长序列,可以引出抽样定理。

连续信号可以进行连续傅里叶变换,而周期信号需要进行傅里叶级数,时域的周期性造成了频域的离散。时域的离散造成频域的周期性。

CTFS 是主值区间的信号,CTFT 是频域的采样。

$$
\begin{array}{l}
X\left(e^{j \Omega T}\right)=\sum_{n=-\infty}^{\infty} x(n) e^{-j \omega n} \\
x(n)=\frac{1}{2 \pi} \int_{-\pi}^{\pi} X\left(e^{j \omega}\right) e^{j \omega n} d \omega
\end{array}
$$


\section{离散傅立叶变换}


若是一个周期序列,那么不是绝对可和的,不能使用 DTFT 。

若是周期为 \(N\) 那么:$$
\tilde{x}(n)=\tilde{x}(n+r N)
$$

希望展开成离散的傅里叶级数:

$$
\tilde{x}(n)=\frac{1}{N} \sum_{k=0}^{N-1} \tilde{X}(k) e^{j \frac{2 \pi}{N} k n}
$$

$$
\begin{aligned}
\sum_{n=0}^{N-1} \tilde{x}(n) e^{-j \frac{2 \pi}{N} r n} &=\frac{1}{N} \sum_{n=0}^{N-1} \sum_{k=0}^{N-1} \tilde{X}(k) e^{j \frac{2 \pi}{N}(k-r) n} \\
&=\sum_{k=0}^{N-1} \tilde{X}(k)\left[\frac{1}{N} \sum_{n=0}^{N-1} e^{j \frac{2 \pi}{N}(k-r) n}\right]
\end{aligned}
$$

$$
\frac{1}{N} \sum_{n=0}^{N-1} e^{j \frac{2 \pi}{N} r n}=\frac{1}{N} \frac{1-e^{j \frac{2 \pi}{N} r N}}{1-e_{}^{j \frac{2 \pi}{N}}}= 1, \text{ when } r = mN , 0 , \text{ else} 
$$

定义变换因子的符号: \(W_N = e^{-j \frac{2\pi}{N}}\) 。那么变换对为:

$$
\begin{array}{l}
\qquad \tilde{X}(k)=D F S[\tilde{x}(n)]=\sum_{n=0}^{N-1} \tilde{x}(n) e^{-j \frac{2 \pi}{N} n k}=\sum_{n=0}^{N-1} \tilde{x}(n) W_{N}^{n k} \\
\tilde{x}(n)=I D F S[\tilde{X}(k)]=\frac{1}{N} \sum_{k=0}^{N-1} \tilde{X}(k) e^{j \frac{2 \pi}{N} n k}=\frac{1}{N} \sum_{k=0}^{N-1} \tilde{X}(k) W_{N}^{-n k}
\end{array}
$$

DFS 可以看作是主值区间的 Z 变换在单位圆的等间隔抽样。

\[\tilde{X}(k) = X(e^{j\omega}) | _{\omega = w\pi k / N}\]

\subsection{频域采样}

频域采样 \(N\) 点,得到的是抽样点为 \(N\) 的周期延拓。可以用来设计滤波器,若是可以无失真回复原序列,那么可以完整表达 \(X(z)\) 和 \(X(e^{j\omega})\) 。

\subsection{内插器公式}

\[\frac{1-\exp{-j\omega N}}{1-\exp{-j\omega}} = \exp{-j\omega\frac{N-1}{2}} \frac{\sin {\omega N / 2}}{\sin {\omega / 2}}\]

\subsection*{DFS}

是对 DTFT 在主值区间的 N 点采样。周期性和离散性是等价的。

\section{离散傅里叶变换}

当只取主值区间时,就可以得到 DFT 。

$$
\begin{array}{l}
\qquad \tilde{X}(k)=D F T[\tilde{x}(n)]=\sum_{n=0}^{N-1} \tilde{x}(n) e^{-j \frac{2 \pi}{N} n k}=\sum_{n=0}^{N-1} \tilde{x}(n) W_{N}^{n k} \\
\tilde{x}(n)=I D F T[\tilde{X}(k)]=\frac{1}{N} \sum_{k=0}^{N-1} \tilde{X}(k) e^{j \frac{2 \pi}{N} n k}=\frac{1}{N} \sum_{k=0}^{N-1} \tilde{X}(k) W_{N}^{-n k}
\end{array}
$$

规定周期信号和主值区间转换的标记方式: \(x[n] = \tilde{x}[n] R_N[n]\) , \(\tilde{x}[n] = x [n \:\mathrm{ mod }\: N] =x[[n]]_N\) 。

\subsection{对偶性}


注意:\[DFT[X(n)] = N \tilde{x((-k))_N R_N(k)}\]

得到:

$$
I D F T [X(k)]=\frac{1}{N} \overline{D F T\left[X^{*}(k)\right]}
$$


\subsection{性质}

\begin{itemize}
    \item 线性
    \item 循环移位 $x_{m}(n)=x((n+m))_{N} R_{N}(n) = W_N^{-mk} \tilde{X}[k]2$
    \item 圆周共轭对称:实序列对应频域圆周共轭对称,虚序列对应频域圆周共轭反对称
    \item 圆周卷积和:点数有关
    \item 相关不满足交换律
\end{itemize}


\section{DFT 快速算法}

几个特点: 

\begin{itemize}
    \item 对称性: \(W_{N}^{nk} * = W_{N}^{-nk}\) 
    \item 周期性: \(W_{N}^{nk} = W_N^{(n + N) k} = W_N^{n (k + N)}\) 
    \item 可约性: \(W_N^{nk} = W_{m N}^{m n k} \)
\end{itemize}

\subsection{戈泽尔算法}

最典型:利用周期性与对称性。

\[X(k)=\sum_{r=0}^{N-1} x(r) W_{N}=W_{N}^{-k N} \sum_{r=0}^{N-1} x(r) W_{N}=\sum_{r=0}^{N-1} x(r) W_{N}^{-k(N-r)}\]

\[y_{k}(n)=x(n) * W_{N}{ }^{-k n} u(n) = \sum_{r=0}^{N-1} x(r) W_{N}^{-k(n-r)} u(n-r)\]

\[\left.y_{k}(n)\right|_{n=N}=\sum_{r=0}^{N-1} x(r) W_{N}^{-k(N-r)} u(N-r)=\sum_{r=0}^{N-1} x(r) W_{N}^{-k(N-r)}=X(k)\]

避免了乘积项 \(W\) 的反复计算。

\subsection{基2-FFT}

适合递归分解,需要分解成特殊基,并且最小子序列无复数运算。

\subsection{线性调频 Z 变换}

新的变换

% End Here

\let\chapname\undefined
\ifx\mainclass\undefined
\end{document}
\fi 