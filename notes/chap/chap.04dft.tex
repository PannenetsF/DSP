\ifx\mainclass\undefined
\documentclass[cn,11pt,chinese,black,simple]{../elegantbook}
% 微分号
\newcommand{\dd}[1]{\mathrm{d}#1}
\newcommand{\pp}[1]{\partial{}#1}

\newcommand{\homep}[1]{{\Large\textbf{Problem #1}}}
\newcommand{\subhomep}[1]{{\large\textbf{SubProblem #1}}}
\usepackage{mathrsfs} 
% FT LT ZT
\newcommand{\ft}[1]{\mathscr{F}[#1]}
\newcommand{\fta}{\xrightarrow{\mathscr{F}}}
\newcommand{\lt}[1]{\mathscr{L}[#1]}
\newcommand{\lta}{\xrightarrow{\mathscr{L}}}
\newcommand{\zt}[1]{\mathscr{Z}[#1]}
\newcommand{\zta}{\xrightarrow{\mathscr{Z}}}
\newcommand{\ztra}{\xrightarrow{\mathscr{Z}^{-1}}}
\newcommand{\dtft}[1]{\text{DTFT}[#1]}
\newcommand{\dtftr}[1]{\text{DTFT}^{-1}[#1]}
\newcommand{\dtfta}{\xrightarrow{\text{DTFT}}}
\newcommand{\where}[2]{\left.#1\right|_{#2}}

\newcommand{\trans}[1]{{T}[#1]}

% 积分求和号

\newcommand{\dsum}{\displaystyle\sum}
\newcommand{\aint}{\int_{-\infty}^{+\infty}}

% 简易图片插入
\newcommand{\qfig}[3][nolabel]{
  \begin{figure}[!htb]
      \centering
      \includegraphics[width=0.4\textwidth]{#2}
      \caption{#3}
      \label{#1}
  \end{figure}
}

% 表格
\renewcommand\arraystretch{1.5}

\usepackage{multicol}

% 日期

\newcommand{\darr}{\underset{\mathop{\uparrow}}}
\begin{document}
\fi 
\def\chapname{04dft}

% Start Here
\chapter{离散傅里叶变化}

数字处理需要对时域进行采样与截断,频域离散化。

从离散傅里叶级数(DFS)开始,逐步介绍离散傅里叶变换(DFT),以及快速算法。

\section{离散傅立叶级数}

周期序列等价于有限长序列,可以引出抽样定理。

连续信号可以进行连续傅里叶变换,而周期信号需要进行傅里叶级数,时域的周期性造成了频域的离散。时域的离散造成频域的周期性。

CTFS 是主值区间的信号,CTFT 是频域的采样。

$$
\begin{array}{l}
X\left(e^{j \Omega T}\right)=\sum_{n=-\infty}^{\infty} x(n) e^{-j \omega n} \\
x(n)=\frac{1}{2 \pi} \int_{-\pi}^{\pi} X\left(e^{j \omega}\right) e^{j \omega n} d \omega
\end{array}
$$


\section{离散傅立叶变换}


若是一个周期序列,那么不是绝对可和的,不能使用 DTFT 。

若是周期为 \(N\) 那么:$$
\tilde{x}(n)=\tilde{x}(n+r N)
$$

希望展开成离散的傅里叶级数:

$$
\tilde{x}(n)=\frac{1}{N} \sum_{k=0}^{N-1} \tilde{X}(k) e^{j \frac{2 \pi}{N} k n}
$$

$$
\begin{aligned}
\sum_{n=0}^{N-1} \tilde{x}(n) e^{-j \frac{2 \pi}{N} r n} &=\frac{1}{N} \sum_{n=0}^{N-1} \sum_{k=0}^{N-1} \tilde{X}(k) e^{j \frac{2 \pi}{N}(k-r) n} \\
&=\sum_{k=0}^{N-1} \tilde{X}(k)\left[\frac{1}{N} \sum_{n=0}^{N-1} e^{j \frac{2 \pi}{N}(k-r) n}\right]
\end{aligned}
$$

$$
\frac{1}{N} \sum_{n=0}^{N-1} e^{j \frac{2 \pi}{N} r n}=\frac{1}{N} \frac{1-e^{j \frac{2 \pi}{N} r N}}{1-e_{}^{j \frac{2 \pi}{N}}}= 1, \text{ when } r = mN , 0 , \text{ else} 
$$

定义变换因子的符号: \(W_N = e^{-j \frac{2\pi}{N}}\) 。那么变换对为:

$$
\begin{array}{l}
\qquad \tilde{X}(k)=D F S[\tilde{x}(n)]=\sum_{n=0}^{N-1} \tilde{x}(n) e^{-j \frac{2 \pi}{N} n k}=\sum_{n=0}^{N-1} \tilde{x}(n) W_{N}^{n k} \\
\tilde{x}(n)=I D F S[\tilde{X}(k)]=\frac{1}{N} \sum_{k=0}^{N-1} \tilde{X}(k) e^{j \frac{2 \pi}{N} n k}=\frac{1}{N} \sum_{k=0}^{N-1} \tilde{X}(k) W_{N}^{-n k}
\end{array}
$$

DFS 可以看作是主值区间的 Z 变换在单位圆的等间隔抽样。

\[\tilde{X}(k) = X(e^{j\omega}) | _{\omega = w\pi k / N}\]

\section{频域采样}

频域采样 \(N\) 点,得到的是抽样点为 \(N\) 的周期延拓。可以用来设计滤波器,若是可以无失真回复原序列,那么可以完整表达 \(X(z)\) 和 \(X(e^{j\omega})\) 。

\section{内插器公式}

\[\frac{1-\exp{-j\omega N}}{1-\exp{-j\omega}} = \exp{-j\omega\frac{N-1}{2}} \frac{\sin {\omega N / 2}}{\sin {\omega / 2}}\]

% End Here

\let\chapname\undefined
\ifx\mainclass\undefined
\end{document}
\fi 