\ifx\mainclass\undefined
\documentclass[cn,11pt,chinese,black,simple]{../elegantbook}
% 微分号
\newcommand{\dd}[1]{\mathrm{d}#1}
\newcommand{\pp}[1]{\partial{}#1}

\newcommand{\homep}[1]{{\Large\textbf{Problem #1}}}
\newcommand{\subhomep}[1]{{\large\textbf{SubProblem #1}}}
\usepackage{mathrsfs} 
% FT LT ZT
\newcommand{\ft}[1]{\mathscr{F}[#1]}
\newcommand{\fta}{\xrightarrow{\mathscr{F}}}
\newcommand{\lt}[1]{\mathscr{L}[#1]}
\newcommand{\lta}{\xrightarrow{\mathscr{L}}}
\newcommand{\zt}[1]{\mathscr{Z}[#1]}
\newcommand{\zta}{\xrightarrow{\mathscr{Z}}}
\newcommand{\ztra}{\xrightarrow{\mathscr{Z}^{-1}}}
\newcommand{\dtft}[1]{\text{DTFT}[#1]}
\newcommand{\dtftr}[1]{\text{DTFT}^{-1}[#1]}
\newcommand{\dtfta}{\xrightarrow{\text{DTFT}}}
\newcommand{\where}[2]{\left.#1\right|_{#2}}

\newcommand{\trans}[1]{{T}[#1]}

% 积分求和号

\newcommand{\dsum}{\displaystyle\sum}
\newcommand{\aint}{\int_{-\infty}^{+\infty}}

% 简易图片插入
\newcommand{\qfig}[3][nolabel]{
  \begin{figure}[!htb]
      \centering
      \includegraphics[width=0.4\textwidth]{#2}
      \caption{#3}
      \label{#1}
  \end{figure}
}

% 表格
\renewcommand\arraystretch{1.5}

\usepackage{multicol}

% 日期

\newcommand{\darr}{\underset{\mathop{\uparrow}}}
\begin{document}
\fi 
\def\chapname{01discretetimeandsys}

% Start Here
\chapter{离散时间信号与系统}

\begin{introduction}
    \item 六大信号
    \item 九种运算
    \item 五类系统
\end{introduction}

\section{离散时间信号}

\subsection{定义}

\begin{definition}
    仅仅在离散时刻点有定义的信号或不连续的时刻给出函数值的函数,通常用集合表示,记作 

    \[x = {x(n)} \leftarrow {x_a{nT}}, n \in \mathbb{Z}\]

    需要注意第一项这里 \(n\) 没有时间的单位,但是没有物理的单位。第二项是采样产生,则代表时间的单位。
\end{definition}

\subsection{基本序列}

\begin{enumerate}
    


    \item 单位冲激序列

\[\delta(n = n_0) = \left\{\begin{aligned}
    0, &{n \neq n_0}\\
    1, &{n = n_0}
\end{aligned}\right.\]


\item 单位阶跃序列

\[
u(n-n_0) = \left\{\begin{aligned}
    0, & n < n_0\\
    1, & n \leq 0
\end{aligned}\right\}
\]

对单位冲激存在累加差分关系:

\[
    \delta(n) = u(n) - u(n-1) \text{ and } u(n) = \sum_{k=-\infty}^n \delta(k)
\]

\item \textbf{窗口序列}

\[
    R_N(n) = G_N(n) = \left\{\begin{aligned}
    1, & 0 \leq n \leq N-1\\
    0, & \text{else}
\end{aligned}\right.
\]

和单位阶跃存在减法关系

\[R_N(n) = u(n) - u(n-N)\]

\item 正余弦序列

\[
x(n) = A \cos(\omega_0 \cdot n + \theta_0)    
\]

注意序列可能不为周期序列,但是仍然称 \(\omega_0\) 为序列的频率。

\item 指数序列

\[x(n) = A \alpha^n\]

\item \textbf{周期序列}

\[x(n) = x(n + N)\]

一般来说,线性的系统还有对应的圆周系统,圆周移位、圆周相关等。


\end{enumerate}

\subsection{基本运算}

\begin{enumerate}
    \item 移位 \[y(n) = x(n-m)\]
    \item 反褶 \[y(n) = x(-n)\]
    \item 和差 \[y(n) = x_1(n) \pm x_2(n)\]
    \item 积商 \[y(n) = x_1(n) \times / \div x_2(n)\]
    \item 累加 \[S(n) = \sum_{k=-\infty}^n x(k)\]
    \item 差分 \begin{itemize}
        \item 前向差分 \[\Delta x(n) = x(n+1) - x(n)\]
        \item 后向差分 \[\nabla x(n) = x(n) - x(n-1)\]
    \end{itemize}
    \item 卷积 \[y(n) = x(n) \otimes h(n) = \sum_{m=-\infty}^{+\infty}x(m)\cdot h(n-m) = \sum_{m=-\infty}^{+\infty}x(n-m)\cdot h(m)\]
    \item 相关 \[R_{xy}(\tau) = \sum_{n=-\infty}^{+\infty} x(n) \cdot y(n-\tau) = x(n) \otimes y(-n)\]
    可以用来检索原有信号的识别与锁定。
    \item 能量 \[E = \sum_{-\infty}^{+\infty} x(n) x^*(n) = \sum_{-\infty}^{+\infty} |x(n)|^2\]
    \item 功率 \[P = \lim_{N\rightarrow \infty} \frac{1}{2 N + 1} \sum_{-N}^N |x(n)|^2\]
\end{enumerate}1



\section{离散时间系统}

\section{线性时不变系统}

\section{线性常系数差分方程}
% End Here

\let\chapname\undefined
\ifx\mainclass\undefined
\end{document}
\fi 