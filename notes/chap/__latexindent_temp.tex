\ifx\mainclass\undefined
\documentclass[cn,11pt,chinese,black,simple]{../elegantbook}
% 微分号
\newcommand{\dd}[1]{\mathrm{d}#1}
\newcommand{\pp}[1]{\partial{}#1}

% FT LT ZT
\newcommand{\dtft}[1]{\text{DTFT}[#1]}
\newcommand{\dtftr}[1]{\text{DTFT}^{-1}[#1]}
\newcommand{\dtfta}{\xrightarrow{\text{DTFT}}}

\newcommand{\where}[1]{\Big|_{#1}}
\newcommand{\abs}[1]{\left| #1 \right|}
\newcommand{\zt}[1]{\mathscr{Z}[#1]}
\newcommand{\ztr}[1]{\mathscr{Z}^{-1}[#1]}
\newcommand{\zta}{\xrightarrow{\mathscr{Z}}} 
\newcommand{\lt}[1]{\mathscr{L}[#1]}
\newcommand{\lta}{\xrightarrow{\mathscr{L}}} 
\newcommand{\ft}[1]{\mathscr{F}[#1]}
\newcommand{\fta}{\xrightarrow{\mathscr{F}}} 
\newcommand{\dsum}{\displaystyle\sum}
\newcommand{\aint}{\int_{-\infty}^{+\infty} }

% 积分求和号

% 简易图片插入
\newcommand{\qfig}[3][nolabel]{
  \begin{figure}[!htb]
      \centering
      \includegraphics[width=0.6\textwidth]{#2}
      \caption{#3}
      \label{#1}
  \end{figure}
}

\usepackage{pdfpages}
\includepdfset{pagecommand=\thispagestyle{plain}}
% 表格
\renewcommand\arraystretch{1.5}

\usepackage{shapepar}
\usepackage{longtable}
% 日期

\begin{document}
\fi 
\def\chapname{03sampandrebuild}

% Start Here
\chapter{信号采样与重构}
% End Here

\begin{introduction}
    \item 数模频率的对应关系,时域采样对频域的影响
    \item 采样信号如何包含连续信号所有信息?如何无失真恢复信号?是否有冗余信息?是否可以进行速率变化?
    \item 离散处理如何等效模拟 LTI 系统?如何提高处理性能?
\end{introduction}

\section{理想周期采样重构}

一般采样都是不可逆的,为了不丢失信息,需要进行约束。

理想采样: \[x_{s}(t)=x_{c}(t) * s(t)=\sum_{-\infty}^{+\infty} x_{c}(n T) \delta(\mathrm{t}-n T)\]

AD 是 CD 的工程近似。

时域 \(s(t) = \dsum_{-\infty}^\infty \delta(t-nT)\) 。数字采样 \(S(\Omega) = \dfrac{2 \pi}{T} \dsum_{-\infty}^\infty \delta(\Omega - n \Omega_0)\) 。

\subsection{整体流程}

采样信号 \[s(t)=\sum_{n=-\infty}^{\infty} \delta(t-n T)\]

调制采样 \[\begin{aligned}
    x_{s}(t) &=x_{c}(t) s(t) \\
    &=x_{c}(t) \sum_{n=-\infty}^{\infty} \delta(t-n T) \\
    &= \sum_{n=-\infty}^{\infty} x_c(nT) \delta(t-n T) 
    \end{aligned}\]

根据 \(\Omega_s = 2\pi / T\) 采样频率 ,其傅立叶变换

\[S(\mathrm{j} \Omega)=\frac{2 \pi}{T} \sum_{k=-\infty}^{\infty} \delta\left(\Omega-k \Omega_{s}\right)\]

那么 \[X_s(j\Omega) = \frac{1}{2\pi} X_c(j\Omega) \otimes S(j\Omega) = \frac{1}{T} \sum_{k=-\infty}^\infty X_c(j\Omega - kj\Omega_s)\] 

数字信号的频谱是模拟频谱的映射,数字的频谱是中心频谱的镜像。

\subsection{采样定理}

平移频谱不交叠。

\subsection{离散和连续 LTI 系统的等效性}

\[Y_{r}(j \Omega)=H_{r}(j \Omega) Y_{s}(j \Omega)=\left.H_{r}(j \Omega) Y\left(e^{j \omega}\right)\right|_{\omega=\Omega T}\]

触底带宽:采样的范围。

\section{抽取和内插}

多速率处理要保证信号的信息不会丢失。抽取是数字域上的采样,内插是数字域上的重构。

\subsection{信号的整倍数采样}

又称降采样,如 \figref{c301} : \[x_D(m) = x(mD)\]  

\qfig[c301]{c301.png}{抽取器或采样压缩器}

存在间隔的冲激: \[\delta_{D}(n)=\frac{1}{D} \sum_{i=0}^{D-1} e^{j \frac{2 \pi}{D} n i}=\left\{\begin{array}{ll}
    1 & n=0, \pm D, \pm 2 D, \ldots \ldots \\
    0 & \text { 其他 }
    \end{array}\right.\]

那么 \(x'(n) = x(n) \delta_D(n)\)  , \(x_D(n) = x(nD) = x'(nD)\) 

其 \(Z\) 变换, \[\begin{aligned}
    X_D(z) &= \sum_{n=-\infty}^\infty x_D(n) z^{-n} \\ 
    &=\sum_{m=-\infty}^{+\infty} x(m) \delta_{D}(m) z^{-m / D} \\
    &=\sum_{m=-\infty}^{+\infty}\left(x(m) \frac{1}{D} \sum_{i=0}^{D-1} e^{j \frac{2 \pi}{D} m i}\right) z^{-m / D} \\
    &=\frac{1}{D} \sum_{i=0}^{D-1} \sum_{m=-\infty}^{+\infty} x(m) e^{j \frac{2 \pi}{D} m i} z^{-m / D} \\
    &=\frac{1}{D} \sum_{i=0}^{D-1} \sum_{m=-\infty}^{+\infty} x(m)\left(z^{\frac{1}{D}} e^{-j \frac{2 \pi}{D} i}\right)^{-m} \\
    &=\frac{1}{D} \sum_{i=0}^{D-1} X\left(z^{\frac{1}{D}} e^{-j \frac{2 \pi}{D} i}\right)
\end{aligned}\]

采样周期变化为原有的 \(D\) 倍:

\[x_{D}(n) \Leftrightarrow X_{D S}(\Omega)=\frac{1}{D T} \sum_{k=-\infty}^{+\infty} X c\left(\Omega-k \frac{\Omega_{0}}{D}\right) \Leftrightarrow X_{D}(\omega)=\frac{1}{D T} \sum_{k=-\infty}^{+\infty} X c\left(\frac{\omega}{D T}-k \frac{2 \pi}{D T}\right)\] 

其对应的模拟滤波器可以这样对待:

\[\begin{aligned}
    X_D(\omega) &= \frac{1}{DT} \sum_{k = -\infty}^\infty X_c(\frac{\omega}{DT} - k \frac{2\pi}{DT}) \\ 
    &= \frac{1}{D} \sum_{i = 0}^{D-1} \frac{1}{T} \sum_{m=-\infty}^\infty X_C (\frac{\omega}{DT} - m \frac{2\pi}{T} - i \frac{2 \pi}{DT}) \\ 
    &= \frac{1}{D} \sum_{i = 0}^{D-1} \frac{1}{T} \sum_{m=-\infty}^\infty X_C (\frac{\omega - 2 \pi i}{DT} - m \frac{2\pi}{T}) \\ 
\end{aligned}\]



\let\chapname\undefined
\ifx\mainclass\undefined
\end{document}
\fi 