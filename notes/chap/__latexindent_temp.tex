\ifx\mainclass\undefined
\documentclass[cn,11pt,chinese,black,simple]{elegantbook}
% 微分号
\newcommand{\dd}[1]{\mathrm{d}#1}
\newcommand{\pp}[1]{\partial{}#1}

% FT LT ZT
\newcommand{\dtft}[1]{\text{DTFT}[#1]}
\newcommand{\dtftr}[1]{\text{DTFT}^{-1}[#1]}
\newcommand{\dtfta}{\xrightarrow{\text{DTFT}}}

\newcommand{\where}[1]{\Big|_{#1}}
\newcommand{\abs}[1]{\left| #1 \right|}
\newcommand{\zt}[1]{\mathscr{Z}[#1]}
\newcommand{\ztr}[1]{\mathscr{Z}^{-1}[#1]}
\newcommand{\zta}{\xrightarrow{\mathscr{Z}}} 
\newcommand{\lt}[1]{\mathscr{L}[#1]}
\newcommand{\lta}{\xrightarrow{\mathscr{L}}} 
\newcommand{\ft}[1]{\mathscr{F}[#1]}
\newcommand{\fta}{\xrightarrow{\mathscr{F}}} 
\newcommand{\dsum}{\displaystyle\sum}
\newcommand{\aint}{\int_{-\infty}^{+\infty} }

% 积分求和号

% 简易图片插入
\newcommand{\qfig}[3][nolabel]{
  \begin{figure}[!htb]
      \centering
      \includegraphics[width=0.6\textwidth]{#2}
      \caption{#3}
      \label{#1}
  \end{figure}
}

\usepackage{pdfpages}
\includepdfset{pagecommand=\thispagestyle{plain}}
% 表格
\renewcommand\arraystretch{1.5}

\usepackage{shapepar}
\usepackage{longtable}
% 日期

\begin{document}
\fi 
\def\chapname{05filter}

% Start Here
\chapter{滤波器设计基础}

\section{滤波器是什么}

\subsection{滤波器基础概念}

LTI 系统的时域卷积和特性造成了频域的滤波效应,主要是指的幅度特性,并且滤波器不会造成新的频率分量。

根据频段内的幅度特性分为四类基本滤波器:低通、高通、带通、带阻。

根据实现方式,也就是冲激的形式:分为 IIR (无限长冲激响应系统)与 FIR (有限长 单位冲激响应)。

\subsection{滤波器设计步骤}


\begin{enumerate}
    \item 按照任务要求,确定滤波器的性能指标
    \item 用一个因果稳定的离散线性时不变系统按照某一准则去逼近所要得到的性能指标可以选用IIR滤波器也可选用FIR滤波器
    \item 利用有限精度,选择合适的系统结构来实现
    \begin{itemize}
        \item 软件实现
        \item 硬件实现
    \end{itemize}
\end{enumerate}

\subsection{性能指标}

增益、衰减都使用分贝来描述。

通带指标:在通带中幅度响应以误差 \(\pm \alpha\) 逼近 \(1\) ,这称作通带容限,或者表示为通带波纹 \(R_p\) ,通道最大衰减。其中 \(\omega_c\) 指通带边缘,其幅度特性为 \(1 - \alpha_1\) 。波纹一般是最大衰减的两倍。

\[
\begin{array}{l}
1-\alpha_{1} \leq\left|H\left(e^{j w}\right)\right| \leq 1+\alpha_{1} \\
R_{P}=-20 \lg \left(\frac{1-\alpha_{1}}{1+\alpha_{1}}\right) \\
\delta_{1}=20 \lg \left|\frac{1}{H\left(e^{j \omega_{c}}\right)}\right|=-20 \lg \left(1-\alpha_{1}\right)=20 \lg \left|\frac{H_{\max }\left(e^{j w}\right)}{H\left(e^{j \omega_{c}}\right)}\right|
\end{array}
\]

阻带指标:阻带中幅度响应以误差 \(\alpha_2\) 逼近 \(0\) ,这叫做阻带容限,阻带波纹为 \(\delta_2\) ,最小衰减为 \(A_s\) 。在 \(\omega_{st}\) 的位置,幅值为 \(\alpha_2\) 。

\[
\begin{array}{l}
0 \leq\left|H\left(e^{j w}\right)\right| \leq \alpha_{2} \\
\delta_{2}=A_{s}=-20 \lg \left(\alpha_{2}\right)
\end{array}
\]

过渡带一般指一段带宽 \(|\omega_{st} - \omega_c|\) 一般对其宽度有要求,其幅度特性一般不要求。


\section{滤波器结构}

根据数据流进行系统的搭建。

转置定理,输入输出及其支路方向逆转可以,再将输入输出交换,结论仍成立。

\subsection{IIR 基本结构}

{直接 I 型} : 输入移位 + 输出移位。

直接 II 型 : 交换次序,合并支路。

直接I型和II型共同缺点是结构系数与系统函数的零、极点关系不明显,对滤波器性能控制不直接。

系统的极点在某些临界条件下对系数的变化过于灵敏,从而使系统频率响应对系数的变化过于灵敏,对有限精度(有限字长)运算容易出现不稳定或较大误差。

级联型 : 便于准确实现滤波器零、极点,调整滤波器频率响应性能与直接II型相同具有最少的存储单元。

由于信号是依次通过级联的各个环节,只要一个环节出现了问题,变影响整个系统。

并联型:单独调整并联子系统的极点,子系统信号无交叉,系统稳定。

但是不能单独调节零点。

\subsection{FIR 基本结构}

直接:卷积和公式

转置:存储单元、计算量最少

级联型:分解为二阶因子形式

频率抽样型:

线性相位型:

\section{模拟域设计}

将 \(S\) 平面转换到 \(Z\) 平面 。

步骤:

\begin{itemize}
    \item 离散指标转换到模拟指标
    \item 模拟原型滤波器来逼近 \(H_c(s)\)
    \item 变换到数字滤波器 \(H(z)\)
    \item 数字滤波器进行等效变换后完成性能验证 \(H_{eff}(s)\)
\end{itemize}

变换条件:

\begin{itemize}
    \item 频率响应对应:\(S\) 平面虚轴需要映射到 \(Z\) 平面单位圆
    \item 因果稳定:\(S\) 左半平面映射到单位圆内部,这样才是一个外圆环
\end{itemize}

主要方式:

\begin{itemize}
    \item 冲激响应不变法
    \item 双线性 Z 变换
\end{itemize}


\subsection{冲激响应不变法}


方法:使数字滤波器的单位冲激响应序列为模拟单位冲激响应的采样。


\[
h(n)=\left.h_{c}(t)\right|_{t=n T}=h_{c}(n T)
\]

\[
\left.H(z)\right|_{z=e^{r}}=\frac{1}{T} \sum_{n=-\infty}^{+\infty} H_{c}\left(s-j k \Omega_{s}\right)=\frac{1}{T} \sum_{n=-\infty}^{+\infty} H_{c}\left(s-j \frac{2 \pi k}{T}\right)
\]


原型:

\[
H_{c}(s)=\sum_{k=1}^{N} \frac{A_{k}}{s-s_{k}}
\]


将模拟滤波器分解为极点并联形式,不保证零点对应。
采样时间太小时增益过高,需要进行一定衰减,如在系统函数乘以周期。
模拟原型滤波器时域通常都无法理想带限,高频采样可以减少混叠。



\[
H_{e f f}(j \Omega)=\left\{\begin{array}{cc}
\left.H\left(e^{j \omega}\right)\right|_{\omega=\Omega T}, & |\Omega|<\pi / T \\
0, & |\Omega| \geq \pi / T
\end{array}\right.
\]

改换到阶跃响应不变法可以减少混叠。


\subsection{双线性 Z 变换}

冲激响应不变法和阶跃响应不变法由于使用了时域上的
采样原理,所以不可避免的产生了频率响应混叠失真。

这是一种单值映射。

\[
s=\frac{1-z^{-1}}{1+z^{-1}}
\]

工程中有 \[
    \Omega=c^{*} \operatorname{tg}(\omega / 2)
    \]

那么

\[
H(z)=\left.H_{c}(s)\right|_{s=c \frac{1-z^{-1}}{1+z^{-1}}}
\]

变换不影响因果性与稳定性:

\[
\begin{aligned}
z & =\frac{c+s}{c-s} \\
|z| &=\sqrt{(c+\sigma)^{2}+\Omega^{2}} / \sqrt{(c-\sigma)^{2}+\Omega^{2}} \\ 
s &= \sigma + j \Omega
\end{aligned}
\]

\begin{itemize}
    \item \(\sigma < 0\) ,  \(|z| > 1\) 
    \item \(\sigma > 0\) ,  \(|z| < 1\) 
    \item \(\sigma = 0\) ,  \(|z| = 1\) 
\end{itemize}


当时频率直接出现了畸变,并且丧失了线性相位特性。

\subsubsection{频率预畸变}

计算在通带、阻带的边缘频率,设计在这些位置满足指标的模拟原型。其中 \( 2 / T_D = c\) 。

\[
    \begin{array}{l}
        \Omega_{D}=\frac{2}{T_{d}} \tan \left(\frac{\omega_{P}}{2}\right) \\
        \Omega_{s}=\frac{2}{T_{d}} \tan \left(\frac{\omega_{s}}{2}\right)
        \end{array}\]

\subsection{数字频带变换}

通过映射转换关系:将数字低通有理系统\(z\)变成因果稳定的有理系统\(Z\),要求:

\begin{itemize}
    \item 映射是有理函数
    \item 单位圆相互对应
    \item 单位圆内部相互对应
\end{itemize}

\[
    H_{d}(Z)=\left.H_{L}(z)\right|_{z^{-1}=G\left(Z^{-1}\right)}\]

\(z, Z\) 对应 \(\theta, \omega\) :

\[
e^{-j \theta}=\left|G\left(e^{-j \omega}\right)\right| e^{j \arg \left[G\left(e^{-j \omega}\right)\right]}
\]

\section{FIR 最优化准则} 

严格线性相位!

两种准则:

\begin{itemize}
    \item 均方误差最小:无偏估计,主要介绍
    \item 最大误差最小
\end{itemize}

\subsection{矩形窗}

矩形窗设计法满足最小均方误差准则,\(\alpha = (N - 1) / 2\) 。

若有线性相位约束,窗函数必须保证所截取 \(h(n)\) 是对称的。

过渡带带宽正比于主瓣宽度 \(4 \pi / N\)。截止频率位于 \((\omega_p + \omega_{st}) / 2\) 。

但是矩形窗截断造成肩峰为 \(8.95 \%\),则阻带最小衰减为 -21db,在工程上远不够。

\[
w(n)=R_{N}(n) \quad W_{R}\left(e^{j w}\right)=W_{R}(w) e^{=j\left(\frac{N-1}{2}\right) w}
\]

\subsection{三角窗}

\[
w(n)=\left\{\begin{array}{l}
\frac{2 n}{N-1}, 0 \leq n \leq \frac{N-1}{2} \\
2-\frac{2 n}{N-1}, \frac{N-1}{2}<n \leq N-1
\end{array}\right.
\]

\[
W\left(e^{j w}\right)=\frac{2}{N-1}\left[\frac{\sin \left[\left(\frac{N-1}{4}\right) w\right]}{\sin \left(\frac{w}{2}\right)}\right]^{2} e^{-j\left(\frac{N-1}{2}\right) w} \approx \frac{2}{N}\left[\frac{\sin \left(\frac{N w}{4}\right)}{\sin \left(\frac{w}{2}\right)}\right]^{2} e^{-j\left(\frac{N-1}{2}\right) w}
\]


主瓣宽度 \(8 \pi / M\) ,突出主瓣。

\subsection{Hanning}

对次瓣进行衰减,主瓣宽度为 \(8 \pi / N\) ,衰减为 \(44 dB\) 。

\[
w(n)=\frac{1}{2}\left[1-\cos \left(\frac{2 \pi n}{N-1}\right)\right] R_{N}(n)
\]

\[
\begin{array}{l}
W\left(e^{j w}\right)=D T F T \quad[w(n)] \\
=\left\{0.5 W_{R}(w)+0.25\left[W_{R}\left(w-\frac{2 \pi}{N-1}\right)+W_{R}\left(w+\frac{2 \pi}{N-1}\right)\right]\right\} e^{-j\left(\frac{N-1}{2}\right) w} \\
=W(w) e^{-j\left(\frac{N-1}{2}\right) w}
\end{array}
\]

\subsection{Hamming}

\(53 dB\) 。

\[
w(n)=\left[0.54-0.46 \cos \left(\frac{2 \pi n}{N-1}\right)\right] R_{R}(n)
\]

\[
\begin{aligned}
W(w)=& 0.54 W_{R}(w)+0.23\left[W_{R}\left(w-\frac{2 \pi}{N-1}\right)+W_{R}\left(w+\frac{2 \pi}{N-1}\right)\right] \\
& \approx 0.54 W_{R}(w)+0.23\left[W_{R}\left(w-\frac{2 \pi}{N}\right)+W_{R}\left(w+\frac{2 \pi}{N}\right)\right],(\text { 当 } N>1)
\end{aligned}
\]

\subsection{Blackman}

\(74 dB\)  \(12 \pi / N\) 

\[
w(n)=\left[0.42-0.5 \cos \left(\frac{2 \pi n}{N-1}\right)+0.08 \cos \left(\frac{4 \pi n}{N-1}\right)\right] R_{N}(n)
\]

\[
\begin{array}{l}
W(w)=0.42 W_{R}(w) \\
+0.25\left[W_{R}\left(w-\frac{2 \pi}{N-1}\right)+W_{R}\left(w+\frac{2 \pi}{N-1}\right)\right]+0.04\left[W_{R}\left(w-\frac{4 \pi}{N-1}\right)+W_{R}\left(w+\frac{4 \pi}{N-1}\right)\right]
\end{array}
\]

\subsection{Kaiser}

\[
w(n)=\frac{I_{0}\left(\beta \sqrt{1-\left(1-\frac{2 n}{N-1}\right)^{2}}\right)}{I_{0}(\beta)}
\]

\(\beta\) 越大,过渡越大,衰减越大。



% End Here

\let\chapname\undefined
\ifx\mainclass\undefined
\end{document}
\fi 