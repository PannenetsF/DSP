\ifx\mainclass\undefined
\documentclass[cn,11pt,chinese,black,simple]{../elegantbook}
% 微分号
\newcommand{\dd}[1]{\mathrm{d}#1}
\newcommand{\pp}[1]{\partial{}#1}

% FT LT ZT
\newcommand{\dtft}[1]{\text{DTFT}[#1]}
\newcommand{\dtftr}[1]{\text{DTFT}^{-1}[#1]}
\newcommand{\dtfta}{\xrightarrow{\text{DTFT}}}

\newcommand{\where}[1]{\Big|_{#1}}
\newcommand{\abs}[1]{\left| #1 \right|}
\newcommand{\zt}[1]{\mathscr{Z}[#1]}
\newcommand{\ztr}[1]{\mathscr{Z}^{-1}[#1]}
\newcommand{\zta}{\xrightarrow{\mathscr{Z}}} 
\newcommand{\lt}[1]{\mathscr{L}[#1]}
\newcommand{\lta}{\xrightarrow{\mathscr{L}}} 
\newcommand{\ft}[1]{\mathscr{F}[#1]}
\newcommand{\fta}{\xrightarrow{\mathscr{F}}} 
\newcommand{\dsum}{\displaystyle\sum}
\newcommand{\aint}{\int_{-\infty}^{+\infty} }

% 积分求和号

% 简易图片插入
\newcommand{\qfig}[3][nolabel]{
  \begin{figure}[!htb]
      \centering
      \includegraphics[width=0.6\textwidth]{#2}
      \caption{#3}
      \label{#1}
  \end{figure}
}

\usepackage{pdfpages}
\includepdfset{pagecommand=\thispagestyle{plain}}
% 表格
\renewcommand\arraystretch{1.5}

\usepackage{shapepar}
\usepackage{longtable}
% 日期

\begin{document}
\fi 
\def\chapname{03sampandrebuild}

% Start Here
\chapter{信号采样与重构}
% End Here

\begin{introduction}
    \item 数模频率的对应关系,时域采样对频域的影响
    \item 采样信号如何包含连续信号所有信息?如何无失真恢复信号?是否有冗余信息?是否可以进行速率变化?
    \item 离散处理如何等效模拟 LTI 系统?如何提高处理性能?
\end{introduction}

\section{理想周期采样重构}

一般采样都是不可逆的,为了不丢失信息,需要进行约束。

理想采样: \[x_{s}(t)=x_{c}(t) * s(t)=\sum_{-\infty}^{+\infty} x_{c}(n T) \delta(\mathrm{t}-n T)\]

AD 是 CD 的工程近似。

时域 \(s(t) = \dsum_{-\infty}^\infty \delta(t-nT)\) 。数字采样 \(S(\Omega) = \dfrac{2 \pi}{T} \dsum_{-\infty}^\infty \delta(\Omega - n \Omega_0)\) 。

\subsection{整体流程}

采样信号 \[s(t)=\sum_{n=-\infty}^{\infty} \delta(t-n T)\]

调制采样 \[\begin{aligned}
    x_{s}(t) &=x_{c}(t) s(t) \\
    &=x_{c}(t) \sum_{n=-\infty}^{\infty} \delta(t-n T) \\
    &= \sum_{n=-\infty}^{\infty} x_c(nT) \delta(t-n T) 
    \end{aligned}\]

根据 \(\Omega_s = 2\pi / T\) 采样频率 ,其傅立叶变换

\[S(\mathrm{j} \Omega)=\frac{2 \pi}{T} \sum_{k=-\infty}^{\infty} \delta\left(\Omega-k \Omega_{s}\right)\]

那么 \[X_s(j\Omega) = \frac{1}{2\pi} X_c(j\Omega) \otimes S(j\Omega) = \frac{1}{T} \sum_{k=-\infty}^\infty X_c(j\Omega - kj\Omega_s)\] 

数字信号的频谱是模拟频谱的映射,数字的频谱是中心频谱的镜像。

\subsection{采样定理}

平移频谱不交叠。



\let\chapname\undefined
\ifx\mainclass\undefined
\end{document}
\fi 