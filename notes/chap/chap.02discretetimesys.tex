\ifx\mainclass\undefined
\documentclass[cn,11pt,chinese,black,simple]{../elegantbook}
% 微分号
\newcommand{\dd}[1]{\mathrm{d}#1}
\newcommand{\pp}[1]{\partial{}#1}

% FT LT ZT
\newcommand{\dtft}[1]{\text{DTFT}[#1]}
\newcommand{\dtftr}[1]{\text{DTFT}^{-1}[#1]}
\newcommand{\dtfta}{\xrightarrow{\text{DTFT}}}

\newcommand{\where}[1]{\Big|_{#1}}
\newcommand{\abs}[1]{\left| #1 \right|}
\newcommand{\zt}[1]{\mathscr{Z}[#1]}
\newcommand{\ztr}[1]{\mathscr{Z}^{-1}[#1]}
\newcommand{\zta}{\xrightarrow{\mathscr{Z}}} 
\newcommand{\lt}[1]{\mathscr{L}[#1]}
\newcommand{\lta}{\xrightarrow{\mathscr{L}}} 
\newcommand{\ft}[1]{\mathscr{F}[#1]}
\newcommand{\fta}{\xrightarrow{\mathscr{F}}} 
\newcommand{\dsum}{\displaystyle\sum}
\newcommand{\aint}{\int_{-\infty}^{+\infty} }

% 积分求和号

% 简易图片插入
\newcommand{\qfig}[3][nolabel]{
  \begin{figure}[!htb]
      \centering
      \includegraphics[width=0.6\textwidth]{#2}
      \caption{#3}
      \label{#1}
  \end{figure}
}

\usepackage{pdfpages}
\includepdfset{pagecommand=\thispagestyle{plain}}
% 表格
\renewcommand\arraystretch{1.5}

\usepackage{shapepar}
\usepackage{longtable}
% 日期

\begin{document}
\fi 
\def\chapname{week02}

% Start Here
\chapter{离散时间系统变换域分析}

\section{离散时间傅里叶变换}

\begin{introduction}
    \item 离散时间傅里叶变换
    \item 基本序列 DTFT
    \item DTFT 性质与定理
    \item 系统函数
\end{introduction}

\begin{definition}[离散时间傅里叶变换]
    DTFT/ 离散时间傅立叶变换\footnote{DFT 是离散傅里叶变换,切勿混淆},应用与非周期信号以及傅里叶频谱的关系。

    正变换:\[\dtft{x(n)} = X(e^{j \omega}) = \sum_{n=-\infty}^{\infty} x(n) e^{-j \omega n}\]

    反变换:\[\dtftr{X(e^{j \omega})} = x(n) = \frac{1}{2 \pi} \int_{-\pi}^{\pi} X(e^{j\omega})e^{j \omega n} \dd{\omega}\]
\end{definition}

DTFT 将离散的序列变换到了一个连续的函数,对于非周期序列可以收敛,是一个关于 \(\omega\) 的周期函数。

在 MATLAB 中, \lstinline{sinc(x)} 表示 \(\sin(x)/x\)

在反变换中,隐藏了关于信号分解的含义:

\[\begin{aligned}
    x(n) &= \frac{1}{2 \pi} \int_{-\pi}^{\pi} X(e^{j \omega}) e^{j \omega n} \dd{\omega}\\
    &= \frac{1}{2 \pi} \sum_{k = -\infty}^{\infty} X(e^{j k \Delta \omega})e^{j k \Delta \omega n}\Delta \omega 
\end{aligned}\]

其中关于 \(n\) 的只有一项,将频谱的面积乘以对应离散时刻的分量。

\subsection{基本序列的 DTFT}

单位冲激序列:
\[\delta(n) \dtfta 1\]

单位常数序列:
\[1 \dtfta \sum_{k=-\infty}^\infty 2 \pi \delta(\omega + 2 k \pi)\]

单位阶跃序列:
\[u(n) \dtfta \frac{1}{1 - e^{-j \omega}} + \sum_{k = -\infty}^\infty = \sum_{k=-\infty}^\infty \pi \delta(\omega + 2 k \pi)\]

单位指数序列:
\[e^{j \omega_0 n} \dtfta \sum_{k=-\infty}^\infty 2 \pi \delta(\omega - \omega_0 + 2 k \pi)\]

矩形窗序列:

\[G_N(n) \dtfta \frac{\sin(\omega N / 2)}{\sin(\omega / 2)}e^{-j \frac{(N-1)\omega}{2}}\]

理想低通滤波器,截止频率为 \(\omega_c\)

\[H_d(e^{j\omega}) = \left\{
\begin{aligned}
    e^{-j\omega \alpha}, & -\omega_c \leq \omega \leq \omega_c \\
    0, &-\pi < \omega < -\omega_c \text{ or } \omega_c < \omega < \pi 
\end{aligned}    
\right\}\]

反变换:

\[
h_d(n) = \frac{1}{2 \pi} \int_{-\omega_c}^{\omega_c}e^{-j\omega \alpha} e^{j\omega n} \dd{\omega} = \frac{\omega_c}{\pi} \cdot \frac{\sin (\omega_c(n-\alpha))}{\omega_c (n-\alpha)}
\]

\subsection{DTFT 主要性质以及定理}

\begin{itemize}
    \item 线性
    \item 时序频移导致频域调制: \(n\) 在频域只是一个相位系数
    \item 时域调制导致频域平移
    \item 时域反褶导致频域反褶
    \item 时域共轭导致频域共轭以及反褶
    \item 时域相乘形成频域卷积:\[x(n)h(n) \dtfta \frac{1}{2 \pi} \int_{-\pi}^\pi X(e^{j\theta})H(e^{j(\omega-\theta)}) \dd{\theta}\]
    \item 时域卷积导致频域相乘:\[x(n) \otimes h(n) \dtfta X(e^{j\omega}) H(e^{i \omega})\]
    \item 线性保泛变换/帕塞瓦尔定理:\[\sum_{n = -\infty}^\infty x(n) y^*(n) = \frac{1}{2\pi} \int_{-\pi}^{\pi} X(e^{j \omega}) Y^*(e^{j\omega}) \dd{\omega}\]
    \[\sum_{n=-\infty}^{\infty} |x(n)|^2 = \frac{1}{2\pi} \int_{-\pi}^{\pi}  |X(e^{j\omega}) \dd{\omega}|\]
\end{itemize}

\subsection{DTFT 对称性}

\begin{definition}[对称序列]
    共轭对称序列:\[x_e(n) = x_e^*(-n)\]

    共轭反对称序列:\[x_o(n) = -x_o^*(-n)\]
\end{definition}

虚实部以及幅相满足:

$\operatorname{Re}\left[x_{e}(n)\right]=\operatorname{Re}\left[x_{e}(-n)\right]$


$\operatorname{Im}\left[x_{e}(n)\right]=-\operatorname{Im}\left[x_{e}(-n)\right] $ 


$\operatorname{Re}\left[x_{o}(n)\right]=-\operatorname{Re}\left[x_{o}(-n)\right]$

$\operatorname{Im}\left[x_{o}(n)\right]=\operatorname{Im}\left[x_{o}(-n)\right]$

\begin{equation}\begin{array}{l}
    \left|x_{e}(n)\right|=\left|x_{e}(-n)\right| \\
    \operatorname{Arg}\left[x_{e}(n)\right]=-\operatorname{Arg}\left[x_{e}(-n)\right]
\end{array}\end{equation}

\begin{equation}\begin{array}{l}
\left|x_{o}(n)\right|=\left|x_{o}(-n)\right| \\
\operatorname{Arg}\left[x_{o}(n)\right]=\pi-\operatorname{Arg}\left[x_{o}(-n)\right]
\end{array}\end{equation}

进行引申,可以将一个序列进行分解,得到一个对称以及一个反对称序列:

\begin{equation}\begin{array}{l}
    x(n)=x_{e}(n)+x_{o}(n) \\
    x_{e}(n)=\frac{1}{2}\left[x(n)+x^{*}(-n)\right] \\
    x_{o}(n)=\frac{1}{2}\left[x(n)-x^{*}(-n)\right]
\end{array}\end{equation}

对应的频谱:

\begin{equation}\begin{array}{l}
    X\left(e^{j \omega}\right)=X_{e}\left(e^{j \omega}\right)+X_{o}\left(e^{j \omega}\right) \\
    X_{e}\left(e^{j \omega}\right)=\frac{1}{2}\left[X\left(e^{j \omega}\right)+X^{*}\left(e^{-j \omega}\right)\right] \\
    X_{o}\left(e^{j \omega}\right)=\frac{1}{2}\left[X\left(e^{j \omega}\right)-X^{*}\left(e^{-j \omega}\right)\right]
\end{array}\end{equation}

\begin{equation}\begin{array}{l} 
    \text {DTFT[} \left.x_{e}(n)\right]=\frac{1}{2}\left[X\left(e^{j \omega}\right)+X^{*}\left(e^{j \omega}\right)\right]=\operatorname{Re}\left[X\left(e^{j \omega}\right)\right] \\
    \text {DTFT} \left[x_{o}(n)\right]=\frac{1}{2}\left[X\left(e^{j \omega}\right)-X^{*}\left(e^{j \omega}\right)\right]=j \operatorname{Im}\left[X\left(e^{j \omega}\right)\right]
\end{array}\end{equation}

推论:实序列傅立叶变换是共轭对称的,即实部是偶对称,虚部是奇对称;幅度是偶对称,幅角是奇对称。 % TODO: read 

\section{Z 变换及其反变换}

\begin{definition}[z 变换]
    \[X(z) = \sum_{n = -\infty}^\infty x(n)z^{-n}\] 
    称 \(x(n)\) 为 \(x(n)\) 的 Z 变换,可以记为:
    \[\zt{x(n)} = X(z)\]

    注意,幂级数收敛时变换才有意义,需要标注收敛域。
\end{definition}

由于存在衰减,可以变换的范围比 DTFT 更大,存在 Z 变换不一定存在 DTFT 。

Z 变换的充分必要条件是级数绝对可和。

\subsection{收敛域}

对于有限长的序列,只要每一项有界,那么级数收敛,且至少包括有限的 z 平面(不包括 \(z = 0\))。对与和原点关系进行判断。

右边序列(向右趋近无穷),存在一个最小的收敛半径,半径之外全部收敛。无左半轴分量,为因果序列。

左边序列,存在一个最大的半径,之内全部收敛。无右半轴分量,为反因果序列。

双边序列若收敛,必在环状区域收敛。

需要注意,不同的序列其Z变换的数学表达式可以完全一致。因此需要给出对应的收敛区间。

\subsection{性质}

\begin{itemize}
    \item 线性:时域不重合时收敛域为交集,其他情况可能消减零极点产生扩大现象。
    \item 序列移位
    \item 尺度变化
    \item 线性加权/Z 域求导
    \begin{equation}Z[n x(n)]=-z \frac{d X(z)}{d z} \quad R_{x-}<|z|<R_{x+}\end{equation}
    \begin{equation}Z\left[n^{m} x(n)\right]=\left(-z \frac{d}{d z}\right)^{m}[X(z)] \quad R_{x-}<|z|<R_{x+}\end{equation}
    \item 共轭序列
    \item 序列反褶
    \begin{equation}Z[x(-n)]=X\left(\frac{1}{z}\right) \quad R_{x+}^{-1}<|z|<R_{x-}^{-1}\end{equation}
    \item 初值定理(因果序列):
    \[\text{ for } x(n) = x(n) u(n) \text{ get } \lim_{z \rightarrow \infty} X(z) = x(0)\]
    \item 终值定理:
    \[\lim_{z \rightarrow 1} (z-1) X(z) = \lim_{n \rightarrow \infty} x(n)\]
    \item 时域卷积
    \[\begin{aligned}
        y(n) &= x(n) \otimes h(n) \\ 
        Y(z) &= \zt{y(n)} = X(z) H(z) 
    \end{aligned}\]
    \item Z 域复卷积
    \[
    \begin{aligned}
        y(n) &= x(n) h(n) \\ 
        Y(z) &= \frac{1}{2 \pi j} \oint_c X(\frac{z}{v}) H(v) v^{-1} \dd{v} = \frac{1}{2 \pi j} \oint_c X)v_ H(\frac{z}{v}) v^{-1} \dd{v}\\
        &\text{where } R_{x-} R_{h-} < |z| < R_{x+} R_{h+}
    \end{aligned}    
    \]
    \item 周期卷积:将复卷积转换为数学形式明显的形式(围线积分半径固定)
    \[
    \begin{aligned}
        & v = \rho e^{j \theta} , z = r e^{j \omega}  \\ 
    Y(r e^{j \omega}) &= \frac{1}{2 \pi j} \oint_c H(\rho e^{j \theta}) X(\frac{r}{\rho} e^{j (\omega - \theta)}) \frac{\dd{(\rho e^{j\theta})}}{\rho e^{j\theta}} \\ 
    &= \frac{1}{2\pi} \int_{-\pi}^{\pi} H(e^{j \theta}) X(e^{j(\omega-\theta)}) \dd{\theta}
    \end{aligned}
    \]
    \item  帕塞瓦尔定理
    \[\sum_{n = -\infty} ^{\infty} |x(n)|^2 = \frac{1}{2 \pi}\int_{-\pi} ^\pi |X(e^{j\omega} )|^2 \dd{\omega}\]
\end{itemize}


\subsection{Z 反变换}

\begin{definition}[Z 反变换]
    从给定的 Z 变换及其收敛域中还原处原始序列的过程叫做 Z 反变换。
    \[x(n) = \mathscr{Z}^{-1}[X(z)]\]

    实质是求 \(X(z)\) 的幂级数展开,通常使用长除法,部分分式,留数(危险积分法)。
\end{definition}

\subsection*{部分分式法(重点)}

在实际应用中,根据极点进行分解,一般 \(X(z)\) 是 \(z\) 的有理分式也就是

\[X(z) = \frac{B(z)}{A(z)}, \text{ where A and B are polynomials}\]

那么可以展开为 

\[X(z) = \sum_i X_i(z)\]

\[x(n) = \sum_i \ztr{X_i (z)}\]

\section{系统函数与频率相应}

LTI 系统可以由其单位冲激相应完整表示,\(H(z)\) 称为传递函数或者系统函数,描述了线性时不变系统的特点:

\[H(z) = \frac{Y(z)}{X(z)}\] 

\subsection{系统函数和因果性与稳定性}

线性时不变系统稳定的充要条件是冲激响应绝对可和,也就是收敛域包括单位圆:

\[\sum |h(-n)| = \sum |h(-n) z^{-n}| < \infty \]

系统因果的充要条件是开放收敛域,即为 \(|z| \ge 1\) ,极点全部在单位圆内部。

但是系统函数不能处理非零状态的系统响应,要引入单边 Z 变换进行处理。

\subsection{频率响应}

根据 DTFT 开展,得到 

\begin{itemize}
    \item 单位圆上系统函数是频率响应
    \item 传递函数存在,但是频率响应可能存在
    \item LTI 不产生新的频率分量
\end{itemize}


LTI 系统对输入信号可以看作是对一段频率划分的多个复指数分量信号,是对每个分量响应的和。

数字频率存在天花板.

除比例常数K以外,系统函数完全由它的全部零点、极点来确定。

\subsection{幅度响应}

原点处的零极点对幅度无影响,并可以考虑峰谷点以及零极点,远离零极点则会平缓。

\subsection{相位响应}

是对特定的频率进行相位变换。

原点处零极点分别对应超前与滞后。靠近单位圆的零极点会造成较大的群延迟。远离零极点的区域会比较平坦。

\section{有理 LTI 系统幅相特性}

典型的 LTI 系统有,其作用是,

有理系统的幅相特性存在响应的制约。幅度特性已知,那么相位特性仅有有限选择;零极点个数与阶数和相位特性已知,那么也只有有限的幅度特性也是有限选择。若是在最小相位\footnote{最小相位指的是在零极点均在单位圆内,对幅度造成的波动最小}中,一一对应。

\subsection{幅度特性约束下的系统函数}

给定了频率响应的幅度平方特性下:

\[|H(e^{j\omega})|^2 = H^*(e^{j\omega}) H(e^{j\omega}) = H(z)H^*(\frac{1}{z^*}) = C(z), \text{ where } z = e^{j\omega}\] 


两部分在\(j\omega\)的视角共轭,而在\(z\)的角度是在单位圆内外分布的,零极点以单位圆的镜像映射。

为了使得差分方程为实系数,那么零极点是共轭对或者实数。

\subsection{全通系统}

若系统对所有的频率分量的幅度响应均为非零恒定值,则该系统即为全通系统。那么零极点是镜像映射。

幅度特性不变,因此需要了解其相位特性。

应用:相位均衡器;因果稳定系统的分解分解为全通与最小相位系统的级联;对非稳定系统,可以保证幅度特性不变.

\subsection{最小相位系统}

在幅度特性之外需要确定相位特性,零极点全在单位圆内,以及单位圆外的零点\footnote{没有极点,是因为这是 LTI 系统}映射。

特征:最快响应!
\begin{itemize}
    \item 最小相位延迟
    \item 最小群延迟
    \item 最小能量延迟
\end{itemize}

乘以全通因子后,能量一定会滞后。

\subsection{逆系统与失真补偿}

逆系统:与原系统级联结果为 \(1\) : \(H_i(z) = 1 / H(e^{j\omega})\) 。最小相位系统存在一个因果稳定的逆系统的系统。

若一个信号已经被某个不合要求的频率响应的LTI系统所失真,那么可以用一个补偿系统来处理这个失真了的信号。
此时幅度失真和相位失真不可能被同时补偿;幅度失真可以利用其最小相位因子完全补偿。

\subsection{线性相位系统}

即相位不失真系统。

其频率响应: \(H(e^{j\omega}) = e^{-j\omega \alpha}\) ,相位为 \(-\omega \alpha\) 群延时为 \(\alpha\) 。


\section{线性相位分解}

线性系统代表一个零点对应四个零点。

% End Here

\let\chapname\undefined
\ifx\mainclass\undefined
\end{document}
\fi 