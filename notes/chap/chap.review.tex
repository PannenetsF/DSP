\ifx\mainclass\undefined
\documentclass[cn,11pt,chinese,black,simple]{elegantbook}
% 微分号
\newcommand{\dd}[1]{\mathrm{d}#1}
\newcommand{\pp}[1]{\partial{}#1}

% FT LT ZT
\newcommand{\dtft}[1]{\text{DTFT}[#1]}
\newcommand{\dtftr}[1]{\text{DTFT}^{-1}[#1]}
\newcommand{\dtfta}{\xrightarrow{\text{DTFT}}}

\newcommand{\where}[1]{\Big|_{#1}}
\newcommand{\abs}[1]{\left| #1 \right|}
\newcommand{\zt}[1]{\mathscr{Z}[#1]}
\newcommand{\ztr}[1]{\mathscr{Z}^{-1}[#1]}
\newcommand{\zta}{\xrightarrow{\mathscr{Z}}} 
\newcommand{\lt}[1]{\mathscr{L}[#1]}
\newcommand{\lta}{\xrightarrow{\mathscr{L}}} 
\newcommand{\ft}[1]{\mathscr{F}[#1]}
\newcommand{\fta}{\xrightarrow{\mathscr{F}}} 
\newcommand{\dsum}{\displaystyle\sum}
\newcommand{\aint}{\int_{-\infty}^{+\infty} }

% 积分求和号

% 简易图片插入
\newcommand{\qfig}[3][nolabel]{
  \begin{figure}[!htb]
      \centering
      \includegraphics[width=0.6\textwidth]{#2}
      \caption{#3}
      \label{#1}
  \end{figure}
}

\usepackage{pdfpages}
\includepdfset{pagecommand=\thispagestyle{plain}}
% 表格
\renewcommand\arraystretch{1.5}

\usepackage{shapepar}
\usepackage{longtable}
% 日期

\begin{document}
\fi 
\def\chapname{review}



% Start Here

\chapter{离散信号与系统}

\section{因果性、记忆性}

是否用到了 \(x[n]\) 的未来值/过去值,而不是其他可计算的值。

\section{LTI 系统}

既是线性系统,又是时不变系统,称为LTI系统。其\textbf{充要条件}是 \(y[n] = x[n] \otimes h[n]\) 。

\subsection{因果系统}

\(h[n] = h[n] u[n]\) 

\subsection{稳定系统}

\[
\sum_{-\infty}^{+\infty} |h(n)| =M<+\infty
\]

\subsection{特征频率与 LTI 系统}

若是有一个无限长的指数信号,那么有一个单频信号:

2.27

\[
    \left[e^{j \omega_{0} n}\right]\rightarrow \sum_{k=-\infty}^{+\infty} 2 \pi \delta\left(\omega-\omega_{0}+2 k \pi\right)
\]

但是若是有限长,那么就有引入除去 \(\omega_0\) 的分量,因此对于一个 LTI 系统来说,放大 \(e^{j \omega_0 n}\) 和 \(e^{j \omega_0 n} u[n]\) 需要的系统函数是不一样的。

\section{差分方程的阶数}

输出 \(y[n-i]\) 最高值和最低值 \(i\) 的差值。

LCCDE = linear constant-coefficient difference equation .



\chapter{DTFT 等变换}

\section{变换共轭性质}

具有普适性。

\[
\begin{array}{c}
\mathcal{Z}\left[x^{*}[n]\right]=\sum_{n=-\infty}^{\infty} X^{*}[n] z^{-n}=\left(\sum_{n=-\infty}^{\infty} x[n]\left(z^{*}\right)^{-n}\right)^{*}=X^{*}\left(z^{*}\right) \\
\mathcal{Z}[x[-n]]=\sum_{n=-\infty}^{\infty} x[-n] z^{-n}=\sum_{n=-\infty}^{\infty} x[n]\left(z^{-1}\right)^{-n}=X\left(z^{-1}\right) \\
\mathcal{Z}[\operatorname{Re}\{x[n]\}]=\mathcal{Z}\left[\frac{x[n]+x^{*}[n]}{2}\right]=\frac{1}{2}\left[X(z)+X^{*}\left(z^{*}\right)\right] \\
\mathcal{Z}[\operatorname{Im}\{x[n]\}]=\mathcal{Z}\left[\frac{z[n]-x^{*}[n]}{2 j}\right]=\frac{1}{2 j}\left[X(z)-X^{*}\left(z^{*}\right)\right]
\end{array}
\]

\section{频域阶数}

2.42

若是在原有的系统函数多一个 \(z\) ,说明原来 \(a_0 z^0\) 的位置变成了 \(a_0 z^1\) ,也就是 \(a_n\) 变成了 \(a_{n+1}\) 。 同理 \(z^{-1}\) 对应 \(a_{n-1}\) 。由于使用因果信号, \(z^{-1}\) 的形式更合适。

\section{系统设计}

2.56 

需要一个系统时,可以通过其定义入手,配凑式子。 

% End Here

\let\chapname\undefined
\ifx\mainclass\undefined
\end{document}
\fi 