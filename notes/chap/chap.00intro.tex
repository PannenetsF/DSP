\ifx\mainclass\undefined
\documentclass[cn,11pt,chinese,black,simple]{../elegantbook}
% 微分号
\newcommand{\dd}[1]{\mathrm{d}#1}
\newcommand{\pp}[1]{\partial{}#1}

% FT LT ZT
\newcommand{\dtft}[1]{\text{DTFT}[#1]}
\newcommand{\dtftr}[1]{\text{DTFT}^{-1}[#1]}
\newcommand{\dtfta}{\xrightarrow{\text{DTFT}}}

\newcommand{\where}[1]{\Big|_{#1}}
\newcommand{\abs}[1]{\left| #1 \right|}
\newcommand{\zt}[1]{\mathscr{Z}[#1]}
\newcommand{\ztr}[1]{\mathscr{Z}^{-1}[#1]}
\newcommand{\zta}{\xrightarrow{\mathscr{Z}}} 
\newcommand{\lt}[1]{\mathscr{L}[#1]}
\newcommand{\lta}{\xrightarrow{\mathscr{L}}} 
\newcommand{\ft}[1]{\mathscr{F}[#1]}
\newcommand{\fta}{\xrightarrow{\mathscr{F}}} 
\newcommand{\dsum}{\displaystyle\sum}
\newcommand{\aint}{\int_{-\infty}^{+\infty} }

% 积分求和号

% 简易图片插入
\newcommand{\qfig}[3][nolabel]{
  \begin{figure}[!htb]
      \centering
      \includegraphics[width=0.6\textwidth]{#2}
      \caption{#3}
      \label{#1}
  \end{figure}
}

\usepackage{pdfpages}
\includepdfset{pagecommand=\thispagestyle{plain}}
% 表格
\renewcommand\arraystretch{1.5}

\usepackage{shapepar}
\usepackage{longtable}
% 日期

\begin{document}
\fi 
\def\chapname{00intro}

% Start Here
\chapter{绪论}

本课程采用的主要教材为奥本海默《离散时间信号处理》,程佩清《数字信号处理教程》。第一本是经典教材,深入浅出,繁杂的内容对于初学者不友好,题目也很经典,翻译较差(考试基本就是换参数)。第二本脉络比较清楚,便于理解。

本门课程通过将数学抽象转换为物理概念,从工程角度思考理论问题。

平时考核占 20\% - 30\%,其他是期末考试。教辅是李铮,王8艳萍,在新主楼 F518 。

\section{数字信号理论背景}

什么是数字信号?作用是什么?如何分类?信号处理的核心是什么?

信号的作用是探测、揭示与控制。可以分为随机信号以及确定性信号。由于位宽支持的变大,对精度支持变高了。表示运算与变换,对应工程中的滤波压缩、特征提取。核心运算是傅里叶变换。

已经开过信号与系统之后,为什么学习数字信号处理?

\begin{itemize}
    \item 精度极高
    \item 灵活性好
    \item 可靠性强
    \item 容易集成
    \item 时分复用
    \item 多维处理
\end{itemize}

在理论分析的基础上,需要有工程实现的能力。库里 \(\cdot\) 图基以及桑德 \(\cdot\) 图基,将 \(O(N^2)\) 的算法降低到了 \(O(N\log_2 N / 2)\) 。

\section{课程内容与脉络}

信号通过前置滤波器以及 A/D 处理进行 DSP 处理,之后进行 D/A 转换最后通过低通滤波器即可。

本课程分为五章,前三部分是基础理论,之后是工程设计:

\begin{itemize}
    \item 离散信号与系统
    \item 离散系统变换域分析
    \item 连续信号的离散处理
    \item 离散傅里叶变换以及快速算法
    \item 数字滤波器设计
\end{itemize}

% \section{数字信号处理应用实例}
% End Here

\let\chapname\undefined
\ifx\mainclass\undefined
\end{document}
\fi 