\documentclass[lang=cn,11pt,a4paper,cite=authoryear,twocolumn]{elegantpaper}

% 微分号
\newcommand{\dd}[1]{\mathrm{d}#1}
\newcommand{\pp}[1]{\partial{}#1}

\newcommand{\homep}[1]{{\Large\textbf{Problem #1}}}
\newcommand{\subhomep}[1]{{\large\textbf{SubProblem #1}}}
\usepackage{mathrsfs} 
% FT LT ZT
\newcommand{\ft}[1]{\mathscr{F}[#1]}
\newcommand{\fta}{\xrightarrow{\mathscr{F}}}
\newcommand{\lt}[1]{\mathscr{L}[#1]}
\newcommand{\lta}{\xrightarrow{\mathscr{L}}}
\newcommand{\zt}[1]{\mathscr{Z}[#1]}
\newcommand{\zta}{\xrightarrow{\mathscr{Z}}}
\newcommand{\ztra}{\xrightarrow{\mathscr{Z}^{-1}}}
\newcommand{\dtft}[1]{\text{DTFT}[#1]}
\newcommand{\dtftr}[1]{\text{DTFT}^{-1}[#1]}
\newcommand{\dtfta}{\xrightarrow{\text{DTFT}}}
\newcommand{\where}[2]{\left.#1\right|_{#2}}

\newcommand{\trans}[1]{{T}[#1]}

% 积分求和号

\newcommand{\dsum}{\displaystyle\sum}
\newcommand{\aint}{\int_{-\infty}^{+\infty}}

% 简易图片插入
\newcommand{\qfig}[3][nolabel]{
  \begin{figure}[!htb]
      \centering
      \includegraphics[width=0.4\textwidth]{#2}
      \caption{#3}
      \label{#1}
  \end{figure}
}

% 表格
\renewcommand\arraystretch{1.5}

\usepackage{multicol}

% 日期

\newcommand{\darr}{\underset{\mathop{\uparrow}}}

\title{数字信号处理\quad 第七周作业}
\author{范云潜 18373486}
\institute{微电子学院 184111 班}
\date{\zhtoday}

\begin{document}

\maketitle

作业内容:5.12,5.14,5.18,5.22,5.15,5.33,5.59;

% \tableofcontents

\homep{5.12}

\subhomep{a}

极点:\(z=\pm 0.9 j\) 在单位圆内,因此稳定。

\subhomep{b}

单位圆外的因子:\(z=\pm 3\),那么

\[H_{ap} = \frac{1-9z^{-2}}{1-z^{-2}/9}\]

\[H_1(z) = \frac{1+0.2z^{-1}}{1+0.81z^{-2}} (1-\frac{1}{9}z^{-2})\]


\homep{5.14}

\subhomep{a} \(M=10\) 且偶对称,\(\alpha = 5\) 。

\subhomep{b} 偶对称,关于 \(1/2\) 对称,\(\alpha = 1/2\)

\homep{5.18}

需要对反射的零点的 \(z^{-1}\) 系数进行补偿。

\subhomep{a} 

\(H_{min}(z) = \dfrac{2(1-z^{-1}/2)}{1 + z^{-1}/3}\)

\subhomep{b} 

\(H_{min}(z) = \dfrac{3(1+z^{-1}/3)(1-z^{-1}/2)}{z^{-1}(1+z^{-1}/2)}\)

\subhomep{c} 

\(H_{min}(z) = \dfrac{3(1-z^{-1}/3)}{(1-3z^{-1}/4)}\dfrac{(1-z^{-1}/4)}{(1-4z^{-1}/3)}\)

\homep{5.22}

\subhomep{a} 

\(\begin{aligned}
    \frac{Y(z)}{X(z)} &= \frac{1-a^{-1}z^{-1}}{1-az^{-1}} \\
    y[n] - ay[n-1] &= x[n] - \frac{1}{a} x[n-1]
\end{aligned}\)

\subhomep{b} 


极点为 \(z = a\) ,极点需要在单位圆内,即 \(-1 < a < 1\) 。

\subhomep{c}

如\figref{1}。

\qfig[1]{h7p1.png}{收敛域与零极点}

\subhomep{d}

\[\frac{z-a^{-1}}{z-a} \ztra a^nu[n] - a^{-1}a^{n-1}u[n-1]\]

\subhomep{e} 

\[\begin{aligned}
    (\left.H(z)\right|_{z=e^{j\omega}})^2 &= |H(e^{j\omega})|^2 \\
    &= \left| \frac{1-a^{-1}e^{-j\omega}}{1-ae^{-j\omega}} \frac{1-a^{-1}e^{j\omega}}{1-ae^{j\omega}} \right| \\
    &= \frac{1}{|a|^2} \\
\end{aligned}\]

那么 \(|H(z)| = 1/|a|\)

\homep{5.15}

\subhomep{a} 

是,由分类,\(\alpha = 1\) ,\(H(e^{j\omega}) = 2 + e^{j\omega} + 2 e^{-j2\omega} = e^{j\omega} (2e^{j\omega} + 1 + 2e^{-j\omega}) = e^{-j\omega} (1 + 4 \cos \omega )\) 那么,\(\beta = 0\) ,\(A(e^{j\omega}) =  (1 + 4 \cos \omega )\)。

\subhomep{b} 

不是

\subhomep{c}

是,\(\alpha = 1\) ,\(H(e^{j\omega}) = e^{-j\omega}(e^{j\omega} + 3 + e^{-j\omega}) = e^{-j\omega} (3 + 2 \cos \omega)\) ,那么 \(\beta = 0\) , \(A(e^{j\omega}) = 3 + 2 \cos \omega\) 

\subhomep{d}

是,\(\alpha = 1/2\) ,\(H(e^{j\omega}) = e^{j\omega/2} 2 \cos (\omega/2)\) ,那么 \(\beta = 0\) ,\(A(e^{j\omega}) = 2 \cos (\omega/2)\) 。

\subhomep{e}

是,\(\alpha = 1\) ,\(H^{e^{j\omega}} = e^{-j\omega} (e^{j\omega} - e^{-j\omega}) = e^{-j\omega} j 2 \sin \omega\) ,那么,\(\beta = \frac{\pi}{2}\) ,\(A({e^{j\omega}}) = 2 \sin \omega\) 。


\homep{5.33}

\subhomep{a} 

\(X(z) = S(z) - e^{-8a} S(z) / z^8\) ,\(\therefore \: H_1(z) = 1 - \dfrac{e^{-8a}}{z^8}\)

如\figref{2} ,有八重极点与八重根构成的零点。

\qfig[2]{h7p2.png}{收敛域与零极点}

\subhomep{b}

\(H_2(z) = \dfrac{1}{H_1(z)} = \dfrac{1}{1/(ze^a)^8}\) 

\textcircled{1} \(|z| < e^{-a}\) ,非因果,不稳定。

\textcircled{2} \(|z| > e^{-a}\) ,因果,稳定。 

\subhomep{c} 

需要选择因果信号

\(h_2(n) = \left\{\begin{aligned}
    e^{-an},& n = 8 k, k > 0 \\
    0, & \text{else}
\end{aligned}\right.\)

\subhomep{d}

\(\begin{aligned}
    s[n] &= \delta[n] \\
    x[n] &= \delta[n] - e^{-8n} \delta[n-8] \\
    y[n] &= h_2[n] \otimes x[n] \\ 
    &= h_2[n] - e^{-8a}h_2[n-8] \\
    &= \delta[n]
\end{aligned}\)

\homep{5.59}

\subhomep{a} 

窗函数: \(H(e^{j\omega} = \dfrac{1-e^{-jM\omega}}{1-e^{-j\omega}})\) ,那么 \(H_i(e^{j\omega}) = \frac{1-e^{-j\omega}}{1-e^{-jM\omega}}\) 。

利用幂函数展开

\(H_i(e^{j\omega}) = (\sum_{n=0}^\infty (e^{-jM\omega})^n)(1-e^{-j\omega})\) 

那么 

\(h_i[n] = \sum_{k=0}^\infty (\delta[n-kM] + \delta[n-1-kM])\)

\subhomep{b}

\(h_1[n] \dtfta \dfrac{1-e^{-jM\omega q}}{1-e^{-jM\omega}}\)

\(h_2[n] \dtfta 1 - e^{-j\omega}\)

那么 \(H_{all} = h[n] \otimes h_1[n] \otimes h_2[n]\) 

\(H_{all} (e^{j\omega}) = (1 - e^{-jM\omega q})\)

那么 \(h_{all} = \delta[n] - \delta[n-Mq]\) 

那么 \(0 \leq n < Mq\) 

\subhomep{c}

\[\begin{aligned}
    H_2(e^{j\omega}) &= \dfrac{1}{H(e^{j\omega}) H_1(e^{j\omega})}\\
    &= \dfrac{1}{H(e^{j\omega})} \frac{1-e^{-jM\omega}}{1-e^{-jM\omega q}}
\end{aligned}\]

为了收敛,那么无穷处无极点,设 \(H(e^{j\omega})\) 有 \(P\) 个极点, \(Z\) 个极点,又可知 \(H_2(e^{j\omega})\) 有 \(M\) 零点,\(Mq\) 极点,因此 \(M + P < Mq + Z\)

% Start Here

% End Here

\end{document}