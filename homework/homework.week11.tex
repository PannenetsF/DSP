\documentclass[lang=cn,11pt,a4paper,cite=authoryear,twocolumn]{elegantpaper}

% 微分号
\newcommand{\dd}[1]{\mathrm{d}#1}
\newcommand{\pp}[1]{\partial{}#1}

% FT LT ZT
\newcommand{\dtft}[1]{\text{DTFT}[#1]}
\newcommand{\dtftr}[1]{\text{DTFT}^{-1}[#1]}
\newcommand{\dtfta}{\xrightarrow{\text{DTFT}}}

\newcommand{\where}[1]{\Big|_{#1}}
\newcommand{\abs}[1]{\left| #1 \right|}
\newcommand{\zt}[1]{\mathscr{Z}[#1]}
\newcommand{\ztr}[1]{\mathscr{Z}^{-1}[#1]}
\newcommand{\zta}{\xrightarrow{\mathscr{Z}}} 
\newcommand{\lt}[1]{\mathscr{L}[#1]}
\newcommand{\lta}{\xrightarrow{\mathscr{L}}} 
\newcommand{\ft}[1]{\mathscr{F}[#1]}
\newcommand{\fta}{\xrightarrow{\mathscr{F}}} 
\newcommand{\dsum}{\displaystyle\sum}
\newcommand{\aint}{\int_{-\infty}^{+\infty} }

% 积分求和号

% 简易图片插入
\newcommand{\qfig}[3][nolabel]{
  \begin{figure}[!htb]
      \centering
      \includegraphics[width=0.6\textwidth]{#2}
      \caption{#3}
      \label{#1}
  \end{figure}
}

\usepackage{pdfpages}
\includepdfset{pagecommand=\thispagestyle{plain}}
% 表格
\renewcommand\arraystretch{1.5}

\usepackage{shapepar}
\usepackage{longtable}
% 日期


\title{数字信号处理\quad 第十一周作业}
\author{范云潜 18373486}
\institute{微电子学院 184111 班}
\date{\zhtoday}

\begin{document}

\maketitle

作业内容:4.34,4.43,4.49 

% \tableofcontents

\homep{4.34}

\subhomep{a}

易得, \(H(j\Omega) = e^{-j\Omega T / 2}\) ,而这个连续时间的系统函数在不同的 \(T\) 下不唯一。 \(h_c(t) = \delta(t-T/2)\) 。

\subhomep{b} 

系统函数造成的效果是 \(y[n] = x[n+1/2]\) ,那么 \(y[n] = \cos (2.5 \pi n - 0.5 \pi )\) 。

如 \figref{01}。

\qfig[01]{hw11p1.png}{4.34-b}



\homep{4.43}

\subhomep{a}

对 \(f_c(t) \) 进行追踪: \(Y_c(j\Omega) = F_c(j\Omega) H_{aa}(j \Omega) H(j\Omega) \) ,在 \(F_c\) 所在的频段,幅度特性为 \(1\) ,相位需要为 \(\Omega^3 \) ,那么 \((\omega T)^3 = (800 \omega)^3\) , \(H(e^{j\omega}) = e^{j(800 \omega)^3}, \text{ if } |\omega| <= \pi / 2; 0, \text{ else}\) 。

\subhomep{b}

交叠产生的临界: \(400 \pi = 2\pi / T - 800 \pi \) , \(T = 1/600 \) ,那么  \(H(e^{j\omega}) = e^{j(600 \omega)^3}, \text{ if } |\omega| <= \pi / 2; 0, \text{ else}\) 。


\homep{4.49}

\subhomep{a}

分析系统2 ,\(W_c(j\Omega)\) 是三角形函数的卷积。观察 \(w[n]\) 和 \(y[n]\) 实际无区别,之后的转换也是对应的,因此 \(y_1 = y_2\) 。

但是不能恢复,因为信号 \(y_2\) 的处理中截断了。

如\figref{02},\figref{03} 。

\qfig[02]{hw11p2.png}{4.49-1}

\qfig[03]{hw11p3.png}{4.49-2} 

\subhomep{b}

原始信号采样(连续): \(30 \pi\) ,周期为 \(2\pi / T = 80 \pi\)

搬移: \(30 \pi, 50 \pi\) 

离散: \(0.75 \pi, 0.5\pi\)

平方: \(0.5 \pi, 1.5\pi, 2\pi\) 产生了交叠

\subhomep{c} 

此时 \(\Omega_c = 30 \pi\) , \(\Omega_N = 80 \pi\) ,

原始信号: \(30 \pi\)  

立方: \(30\pi, 90\pi\)

离散: \(0.75 \pi, 0.25\pi\) 虽然频带交叠但是没有混叠

不混叠就有可能恢复。

\subhomep{d} 

可知进行降次的操作可以将信号的频谱进行压缩,采样要求降低。

% Start Here

% End Here

\end{document}