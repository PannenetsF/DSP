\documentclass[lang=cn,11pt,a4paper,cite=authoryear]{elegantpaper}

% 微分号
\newcommand{\dd}[1]{\mathrm{d}#1}
\newcommand{\pp}[1]{\partial{}#1}

\newcommand{\homep}[1]{{\Large\textbf{Problem #1}}}
\newcommand{\subhomep}[1]{{\large\textbf{SubProblem #1}}}
\usepackage{mathrsfs} 
% FT LT ZT
\newcommand{\ft}[1]{\mathscr{F}[#1]}
\newcommand{\fta}{\xrightarrow{\mathscr{F}}}
\newcommand{\lt}[1]{\mathscr{L}[#1]}
\newcommand{\lta}{\xrightarrow{\mathscr{L}}}
\newcommand{\zt}[1]{\mathscr{Z}[#1]}
\newcommand{\zta}{\xrightarrow{\mathscr{Z}}}
\newcommand{\ztra}{\xrightarrow{\mathscr{Z}^{-1}}}
\newcommand{\dtft}[1]{\text{DTFT}[#1]}
\newcommand{\dtftr}[1]{\text{DTFT}^{-1}[#1]}
\newcommand{\dtfta}{\xrightarrow{\text{DTFT}}}
\newcommand{\where}[2]{\left.#1\right|_{#2}}

\newcommand{\trans}[1]{{T}[#1]}

% 积分求和号

\newcommand{\dsum}{\displaystyle\sum}
\newcommand{\aint}{\int_{-\infty}^{+\infty}}

% 简易图片插入
\newcommand{\qfig}[3][nolabel]{
  \begin{figure}[!htb]
      \centering
      \includegraphics[width=0.4\textwidth]{#2}
      \caption{#3}
      \label{#1}
  \end{figure}
}

% 表格
\renewcommand\arraystretch{1.5}

\usepackage{multicol}

% 日期

\newcommand{\darr}{\underset{\mathop{\uparrow}}}

\title{数字信号处理实验二:数字滤波器设计}
\author{范云潜 18373486}
\institute{微电子学院 184111 班}
\date{\zhtoday}

\begin{document}

\maketitle


\tableofcontents

\listoffigures

% Start Here

% End Here


\section{实验目的}

本次实验希望通过 MATLAB 提供的几种滤波器及其估计函数设计 IIR 滤波器,以及几种窗函数及其估计函数设计 FIR 滤波器。认识这几种滤波器关注的核心参数,以及这些核心参数对应着以怎样的方式趋近设计指标。

\section{实验原理}

\subsection{IIR 滤波器设计}

这几类经典滤波器设计方法基本流程是一致的:

\begin{enumerate}
    \item 根据设计指标使用滤波器估计函数进行估计,得到本类估计器的核心参数
    \item 根据上一步得到的设计参数,计算响应的滤波器的频域响应参数:零点系数、极点系数
    \item 根据频域响应参数得到其 Z 变换,计算其频响即可
\end{enumerate}


\subsection{FIR 滤波器设计}

FIR 滤波器是基于窗函数的,也就是这个滤波器系统的时域单位冲激响应,得到窗函数后,由于 FIR 不存在极点,因此窗函数直接对应了 Z 变换的各个系数。

本次实验主要通过 Kaiser 窗进行计算,类似 IIR ,同样基于估计函数与对应滤波器函数进行计算。

\section{实验步骤}

\subsection{IIR 滤波器设计}

根据设计指标,填写相应的 \lstinline{fs} 、\lstinline{Wp} 、 \lstinline{Ws} 、 \lstinline{Rp} 、 \lstinline{Rs} ,以上分别对应着采样频率\footnote{实际上,采样频率并未使用到。}、通带频率、阻带频率、通带纹波衰减分贝数、阻带衰减分贝数。

根据以上数据与 \lstinline{buttord} 、\lstinline{cheb1ord} 、 \lstinline{cheb2ord} 、 \lstinline{ellipord} 所需要的不同的参数进行计算,由于估计器得到的参数可能会和所设参数不同,因此需要注意滤波器函数需要传入的函数\footnote{相关函数传入参数以及返回值尽可以通过 \lstinline{help funct} 得到。}。

之后通过 \lstinline{freqz} 等函数观察其频率响应,通过自定义的 \lstinline{co2exp} 函数将涉及到的频率响应的系统响应系数转换为表达式打印出来以及转换成响应的 \LaTeX 代码用于报告书写。

\subsection{FIR 滤波器设计}

基本步骤与 IIR 设计一致,但是 \lstinline{fir1} 和 \lstinline{fir2} 函数要求窗函数的长度比实际响应的长度多一。此外 \lstinline{kaiserord} 进行估计时,是对不同频率区间的目标幅值进行设计,并且会在计算完成后说明计算出的响应实际的阻通类型。

\section{实验代码}


\lstinputlisting[caption={参数转表达式 \lstinline{diff_wav.m}}]{\string"./lab02/co2exp.m\string"}

\lstinputlisting[caption={IIR 设计 \lstinline{diff_wave.m}}]{\string"./lab02/iir.m\string"}

\lstinputlisting[caption={FIR 设计 \lstinline{lab1_ans.m}}]{\string"./lab02/fir.m\string"}

\section{实验结果}

\subsection{IIR 设计}

运行 \lstinline{iir.m} 文件即可。

\begin{lstlisting}[caption={IIR 实验输出}]
n =  8
Wn_get =  0.57413
ans = n = 8 order Butterworth Highpass Filter: 
 (0.003) * z^0 + (-0.027) * z^-1 + (0.095) * z^-2 + (-0.189) * z^-3 + (0.236) * z^-4 + (-0.189) * z^-5 + (0.095) * z^-6 + (-0.027) * z^-7 + (0.003) * z^-8 
----------------------------------------------------------------
(1.000) * z^{0} + (1.179) * z^{-1} + (1.622) * z^{-2} + (1.019) * z^{-3} + (0.591) * z^{-4} + (0.196) * z^{-5} + (0.051) * z^{-6} + (0.007) * z^{-7} + (0.001) * z^{-8} 
ans = n = 5 order Chebyshev I Highpass Filter: 
 (0.008) * z^0 + (-0.040) * z^-1 + (0.079) * z^-2 + (-0.079) * z^-3 + (0.040) * z^-4 + (-0.008) * z^-5 
----------------------------------------------------------------
(1.000) * z^{0} + (2.219) * z^{-1} + (3.002) * z^{-2} + (2.451) * z^{-3} + (1.233) * z^{-4} + (0.311) * z^{-5} 
ans = n = 5 order Chebyshev II Highpass Filter: 
 (0.061) * z^0 + (-0.213) * z^-1 + (0.363) * z^-2 + (-0.363) * z^-3 + (0.213) * z^-4 + (-0.061) * z^-5 
----------------------------------------------------------------
(1.000) * z^{0} + (0.441) * z^{-1} + (0.765) * z^{-2} + (0.130) * z^{-3} + (0.082) * z^{-4} + (0.002) * z^{-5} 
ans = n = 4 order Elliptic Highpass Filter: 
 (0.044) * z^0 + (-0.107) * z^-1 + (0.144) * z^-2 + (-0.107) * z^-3 + (0.044) * z^-4 
----------------------------------------------------------------
(1.000) * z^{0} + (1.544) * z^{-1} + (1.738) * z^{-2} + (1.002) * z^{-3} + (0.311) * z^{-4} 
Press Enter to Save Figures and Continue...
latex_exp = \dfrac{(0.003) \times z^{0} + (-0.027) \times z^{-1} + (0.095) \times z^{-2} + (-0.189) \times z^{-3} + (0.236) \times z^{-4} + (-0.189) \times z^{-5} + (0.095) \times z^{-6} + (-0.027) \times z^{-7} + (0.003) \times z^{-8} }{(1.000) * z^{0} + (1.179) * z^{-1} + (1.622) * z^{-2} + (1.019) * z^{-3} + (0.591) * z^{-4} + (0.196) * z^{-5} + (0.051) * z^{-6} + (0.007) * z^{-7} + (0.001) * z^{-8} }

\dfrac{(0.008) \times z^{0} + (-0.040) \times z^{-1} + (0.079) \times z^{-2} + (-0.079) \times z^{-3} + (0.040) \times z^{-4} + (-0.008) \times z^{-5} }{(1.000) * z^{0} + (2.219) * z^{-1} + (3.002) * z^{-2} + (2.451) * z^{-3} + (1.233) * z^{-4} + (0.311) * z^{-5} }

\dfrac{(0.061) \times z^{0} + (-0.213) \times z^{-1} + (0.363) \times z^{-2} + (-0.363) \times z^{-3} + (0.213) \times z^{-4} + (-0.061) \times z^{-5} }{(1.000) * z^{0} + (0.441) * z^{-1} + (0.765) * z^{-2} + (0.130) * z^{-3} + (0.082) * z^{-4} + (0.002) * z^{-5} }

\dfrac{(0.044) \times z^{0} + (-0.107) \times z^{-1} + (0.144) \times z^{-2} + (-0.107) \times z^{-3} + (0.044) \times z^{-4} }{(1.000) * z^{0} + (1.544) * z^{-1} + (1.738) * z^{-2} + (1.002) * z^{-3} + (0.311) * z^{-4} }
\end{lstlisting}

\subsubsection{Butterworth}

频率响应如 \figref{l2p1} ,阶数为 8 ,系统函数表达式为如 \eqref{eq1} 。

\begin{equation}\label{eq1}
    \dfrac{\begin{aligned}
        & (0.003) \times z^{0} + (-0.027) \times z^{-1} + (0.095) \times z^{-2} + (-0.189) \times z^{-3} + (0.236) \times z^{-4} \\
        & + (-0.189) \times z^{-5} + (0.095) \times z^{-6} + (-0.027) \times z^{-7} + (0.003) \times z^{-8}
    \end{aligned}}{\begin{aligned}
        &(1.000) * z^0 + (1.179) * z^-1 + (1.622) * z^-2 + (1.019) * z^-3 + (0.591) * z^-4 \\
        & + (0.196) * z^{-5} + (0.051) * z^{-6} + (0.007) * z^{-7} + (0.001) * z^{-8} 
    \end{aligned}}
\end{equation}



\qfig[l2p1]{l2p1.png}{Butterworth 滤波器频率响应}

\subsubsection{Chebyshev I}

频率响应如 \figref{l2p2} ,阶数为 5 ,系统函数表达式为如 \eqref{eq2} 。

\begin{equation}\label{eq2}
    \dfrac{\begin{aligned}
        &(0.061) \times z^{0} + (-0.213) \times z^{-1} + (0.363) \times z^{-2} + \\
        &(-0.363) \times z^{-3} + (0.213) \times z^{-4} + (-0.061) \times z^{-5} 
    \end{aligned}}{\begin{aligned}
        &(1.000) * z^{0} + (0.441) * z^{-1} + (0.765) * z^{-2} + (0.130) * z^{-3} \\
        &+ (0.082) * z^{-4} + (0.002) * z^{-5} 
    \end{aligned}}
\end{equation}

\qfig[l2p2]{l2p2.png}{Chebyshev I 滤波器频率响应}


\subsubsection{Chebyshev II}

频率响应如 \figref{l2p3} ,阶数为 5 ,系统函数表达式为如 \eqref{eq3} 。


\begin{equation}\label{eq3}
    \dfrac{\begin{aligned}
        &(0.061) \times z^{0} + (-0.213) \times z^{-1} + (0.363) \times z^{-2} + \\
        &(-0.363) \times z^{-3} + (0.213) \times z^{-4} + (-0.061) \times z^{-5} 
    \end{aligned}}{\begin{aligned}
        &(1.000) * z^{0} + (0.441) * z^{-1} + (0.765) * z^{-2} + (0.130) * z^{-3} \\
        &+ (0.082) * z^{-4} + (0.002) * z^{-5}
    \end{aligned} }
\end{equation}

\qfig[l2p3]{l2p3.png}{Chebyshev II 滤波器频率响应}

\subsubsection{Elliptic}

频率响应如 \figref{l2p4} ,阶数为 4 ,系统函数表达式为如 \eqref{eq4} 。


\begin{equation}\label{eq4}
\dfrac{\begin{aligned}
    &(0.008) \times z^{0} + (-0.040) \times z^{-1} + (0.079) \times z^{-2} \\
    &+ (-0.079) \times z^{-3} + (0.040) \times z^{-4} + (-0.008) \times z^{-5} 
\end{aligned}}{\begin{aligned}
    & (1.000) * z^{0} + (2.219) * z^{-1} + (3.002) * z^{-2} + \\
    &  (2.451) * z^{-3} + (1.233) * z^{-4} + (0.311) * z^{-5} 
\end{aligned}}
\end{equation}

\qfig[l2p4]{l2p4.png}{Elliptic 滤波器频率响应}

\subsection{FIR 设计}

运行 \lstinline{fir.m} 文件即可。

单位冲激响应如\figref{l2p5}, 频率响应如\figref{l2p6} ,阶数为 72。


\qfig[l2p5]{l2p5.png}{FIR 带通滤波器单位冲激响应}

\qfig[l2p6]{l2p6.png}{FIR 带通滤波器频率响应}

\section{结果分析}

\subsection{IIR 设计}

经过查询相关资料,对于经典 IIR 滤波器来说,有以下相应特征:

\begin{itemize}
    \item Butterworth 滤波器提供在 \(\Omega = 0 \text{ or } \infty\) 时的最佳逼近,因此在低频与高频下有着最平坦的特征,并且整体呈现出单调下降的特点,因此不会出现其他形式的滤波器的波动特性,低通高阻特性在极限频点体现较好。
    \item Chebyshev I 类滤波器通过在通带中引入 \lstinline{Rp} 的波纹,将低通特性与理想特性的误差降低,也就是将低通到高阻的过渡变得更加陡峭,此时阻带仍是最平坦的。
    \item Chebyshev II 类滤波器通过在阻带中加入 \lstinline{Rs} 的波纹,将高阻特性与理想特性的误差降低,同样使得低通到高阻的过渡变得更加陡峭,但是此时通带是最平坦的。通带中没有波纹通常是重要优势。
    \item 椭圆滤波器在通带和阻带中均采引入波纹,这样在直观认识上,低通到高阻的过渡陡峭程度会超过前两类滤波器。此外,往往椭圆滤波器有着同样要求下最低的阶数。
\end{itemize}

而将以上已知性质对比得到的滤波器的阶数可以看到 \(8 > 5 = 5 > 4\) ,满足以上性质。

\subsection{FIR 设计}

经过查询相关资料,对于标准频带\footnote{低通、带通、高通和带阻四类滤波器}的 FIR 设计,基于 \lstinline{fir1} 是最经典的方法,可以实现加窗的线性相位 FIR 数字滤波器。参考 \figref{l2p6}, 满足线性相位条件。

% https://ww2.mathworks.cn/help/signal/ug/iir-filter-design.html
% https://ww2.mathworks.cn/help/signal/ug/fir-filter-design.html

\end{document}