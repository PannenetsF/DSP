\documentclass[lang=cn,11pt,a4paper,cite=authoryear,twocolumn]{elegantpaper}

% 微分号
\newcommand{\dd}[1]{\mathrm{d}#1}
\newcommand{\pp}[1]{\partial{}#1}

\newcommand{\homep}[1]{{\Large\textbf{Problem #1}}}
\newcommand{\subhomep}[1]{{\large\textbf{SubProblem #1}}}
\usepackage{mathrsfs} 
% FT LT ZT
\newcommand{\ft}[1]{\mathscr{F}[#1]}
\newcommand{\fta}{\xrightarrow{\mathscr{F}}}
\newcommand{\lt}[1]{\mathscr{L}[#1]}
\newcommand{\lta}{\xrightarrow{\mathscr{L}}}
\newcommand{\zt}[1]{\mathscr{Z}[#1]}
\newcommand{\zta}{\xrightarrow{\mathscr{Z}}}
\newcommand{\ztra}{\xrightarrow{\mathscr{Z}^{-1}}}
\newcommand{\dtft}[1]{\text{DTFT}[#1]}
\newcommand{\dtftr}[1]{\text{DTFT}^{-1}[#1]}
\newcommand{\dtfta}{\xrightarrow{\text{DTFT}}}
\newcommand{\where}[2]{\left.#1\right|_{#2}}

\newcommand{\trans}[1]{{T}[#1]}

% 积分求和号

\newcommand{\dsum}{\displaystyle\sum}
\newcommand{\aint}{\int_{-\infty}^{+\infty}}

% 简易图片插入
\newcommand{\qfig}[3][nolabel]{
  \begin{figure}[!htb]
      \centering
      \includegraphics[width=0.4\textwidth]{#2}
      \caption{#3}
      \label{#1}
  \end{figure}
}

% 表格
\renewcommand\arraystretch{1.5}

\usepackage{multicol}

% 日期

\newcommand{\darr}{\underset{\mathop{\uparrow}}}

\title{数字信号处理\quad 第十周作业}
\author{范云潜 18373486}
\institute{微电子学院 184111 班}
\date{\zhtoday}

\begin{document}

\maketitle

作业内容:4.36, 4.40, 4.53,4.6,4.7,4.24,4.25

% \tableofcontents

\homep{4.36}

\subhomep{a}

\(\omega_C = \Omega_C / T = 0.5 \pi\) 

解得 \(T \leq 1/400 s\) 

\subhomep{b} 

由于降采样而无重叠,频域被拉伸到原来二倍,因此需要将其压缩, \(T' = 2 T = 1 / 200 s\) 

\homep{4.40} 

容易得到,最后的频谱为 \(H(e^{j \omega / L}) X(e^{j\omega})\) ,那么其时域为 \(x[n - 1 / L]\) 

\homep{4.53}

\subhomep{a}

如 \figref{01} , \figref{02} 。

\qfig[01]{h10p1.png}{}
\qfig[02]{h10p2.png}{}

\subhomep{b}

\[\begin{aligned}
    Y_0(e^{j\omega}) &= X(e^{j\omega}) H_0(e^{j\omega}) \\
    &+ X(e^{j(\omega + \pi )}) H_0 (e^{j(\omega + \pi)})
\end{aligned}\]

\subhomep{c} 

\[\begin{aligned}
    Y(e^{j\omega}) &=  H_0(e^{j\omega})[X(e^{j\omega}) H_0(e^{j\omega}) \\
    &+ X(e^{j(\omega + \pi )}) H_0 (e^{j(\omega + \pi)})] \\ 
    &+  H_0(e^{j(\omega+\pi)})[X(e^{j\omega}) H_0(e^{j(\omega+\pi)}) \\
    &+ X(e^{j(\omega + \pi )}) H_0 (e^{j\omega})] \\ 
\end{aligned}\]


% TODO: Why 1/2 and why minus rather than plus

\homep{4.6} 

\subhomep{a}

\[\begin{aligned}
    H_c(\Omega) &= \int_\mathbb{R} h_c(t) e^{-j\Omega t} \dd{t} \\ 
    &= \frac{1}{\alpha + j \Omega}
\end{aligned}\]

其幅度特性如 \figref{03} 。

\qfig[03]{h10p3.png}{}


\subhomep{b}

\[\begin{aligned}
    h_d[n] &= T e^{-a n T} u[n] \\ 
    H_d(e^{j\omega}) &= T \frac{1}{1 - e^{-\alpha T - j \omega}} \\
    |H_d(e^{j\omega})| &= T \frac{1}{(1-k\cos \omega)^2 + k^2 \sin^2 \omega}\\
    & \text{ where } k = e^{-\alpha T} \\ 
\end{aligned}\]

那么最小值是在 \( -\cos \omega \) 最大处,即 \(\omega = \pi \) ,此时幅度为 \(T/(e^{-a T} + 1)^2\) ,当 \(T \rightarrow \infty\) ,最小值为 \(T\) 。

\homep{4.7} 

\subhomep{a} 

\(x_c(t) = s_c(t) + \alpha s_c(t-\tau_\alpha)\) 

\(X_c(j\Omega) =(1 + \alpha e^{-j\Omega \tau_\alpha}) S_c(j\Omega) \)

\subhomep{b}

\[H(e^{j\omega}) = (1 + \alpha e^{-j\Omega \tau_\alpha / T}) \]

\subhomep{c}

\[\begin{aligned}
    h[n] &= \frac{1}{2\pi} \int_{-\pi}^\pi H(e^{j\omega}) e^{j\omega n} \dd{\omega} \\
    &= \frac{1}{2\pi} (\frac{2 j \sin n \pi}{j n} + \alpha\frac{2 j \sin (n - \tau_\alpha/T)}{j (n -\tau_\alpha/T)}) \\ 
    &= \delta(n) + \alpha delta(n - \tau_\alpha/T)
\end{aligned}\]

\homep{4.24}

\[\Omega_c = 2\pi 5 \times 10^3, \Omega_c = \omega_c/T\] 

a) \(\omega_{c1} = \pi, \omega_{c2} = \pi\) 

b) \(\omega_{c1} = 0.5\pi, \omega_{c2} = 0.5\pi\) 

c) \(\omega_{c1} = 0.5\pi, \omega_{c2} = \pi\) 

d) \(\omega_{c1} = \pi, \omega_{c2} = 0.5\pi\) 

如 \figref{04} 。

\qfig[04]{h10p4.png}{}


\homep{4.25}

\subhomep{a}

如 \figref{05} 

\qfig[05]{h10p5.png}{}


\subhomep{b}

要保证出现低通以及相邻波形不重叠的临界:

\[\begin{aligned}
    \Omega_c &\leq \omega_c / T         \\
    2 \pi / T - \Omega_c &\ge \omega_c / T
\end{aligned}\]

解得 \(0.125 10^{-4} s \leq T \leq 0.875 10^{-4} s\)
% Start Here

% End Here

\end{document}