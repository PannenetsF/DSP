\documentclass[lang=cn,11pt,a4paper,cite=authoryear,twocolumn]{elegantpaper}

% 微分号
\newcommand{\dd}[1]{\mathrm{d}#1}
\newcommand{\pp}[1]{\partial{}#1}

% FT LT ZT
\newcommand{\dtft}[1]{\text{DTFT}[#1]}
\newcommand{\dtftr}[1]{\text{DTFT}^{-1}[#1]}
\newcommand{\dtfta}{\xrightarrow{\text{DTFT}}}

\newcommand{\where}[1]{\Big|_{#1}}
\newcommand{\abs}[1]{\left| #1 \right|}
\newcommand{\zt}[1]{\mathscr{Z}[#1]}
\newcommand{\ztr}[1]{\mathscr{Z}^{-1}[#1]}
\newcommand{\zta}{\xrightarrow{\mathscr{Z}}} 
\newcommand{\lt}[1]{\mathscr{L}[#1]}
\newcommand{\lta}{\xrightarrow{\mathscr{L}}} 
\newcommand{\ft}[1]{\mathscr{F}[#1]}
\newcommand{\fta}{\xrightarrow{\mathscr{F}}} 
\newcommand{\dsum}{\displaystyle\sum}
\newcommand{\aint}{\int_{-\infty}^{+\infty} }

% 积分求和号

% 简易图片插入
\newcommand{\qfig}[3][nolabel]{
  \begin{figure}[!htb]
      \centering
      \includegraphics[width=0.6\textwidth]{#2}
      \caption{#3}
      \label{#1}
  \end{figure}
}

\usepackage{pdfpages}
\includepdfset{pagecommand=\thispagestyle{plain}}
% 表格
\renewcommand\arraystretch{1.5}

\usepackage{shapepar}
\usepackage{longtable}
% 日期


\title{数字信号处理\quad 第十六周作业}
\author{范云潜 18373486}
\institute{微电子学院 184111 班}
\date{\zhtoday}

\begin{document}

\maketitle

作业内容:6.19,6.21,6.27,6.36,7.2,7.4,7.22,7.5,7.15,7.16,7.33。

\homep{6.19}


\[\frac{1 + 2 z^{-1} + \frac{1}{4} z^{-2}}{1 + \frac{1}{4}z^{-2}} \cdot \frac{1}{1 - \frac{3}{2}z^{-1} - z^{-2}} \]

其流图如\figref{01}

\qfig[01]{1601.png}{}

\homep{6.21}

\[\begin{aligned}
    h[n] &= e^{j\omega_0 n} u[n] \\ 
    H(z) &= \frac{1}{1-e^{j\omega_0} z^{-1}} = \frac{Y(z)}{X(z)}\\ 
    x[n] &= y[n] - e^{j\omega_0} y[n-1] \\ 
\end{aligned}\]

已知 \(x[n]\) 为实序列,将实部虚部分开,得到:

\[\begin{aligned}
    x[n] &= y_r[n] - \cos \omega_0 y_r[n-1] + \sin \omega_0 y_i[n-1]\\
     0 &= y_i[n] - \cos \omega_0 y_i[n-1] - \sin \omega_0 y_r[n-1]\\
\end{aligned}\]

其流图如\figref{02}

\qfig[02]{1602.png}{}

\homep{6.27}

\subhomep{a}

\[\begin{aligned}
    H(z) &= \sum h[n] z^{-n} \\ 
    H_1(z) &= \sum h[n] (-1)^n z^{-n} = H(-z) \\
    \therefore H_1(e^{j\omega}) &= H(e^{j\omega + j\pi})
\end{aligned}\]

频响如\figref{03}

\qfig[03]{1603.png}{}


\subhomep{b}

将所有带有 \(z^{-1}\) 的支路修改为 \(-z^{-1}\) 。

\subhomep{c}

流图如\figref{04}

\qfig[04]{1604.png}{}

\homep{6.36}

\subhomep{a}

转置如 \figref{05} 

\qfig[05]{1605.png}{}


\subhomep{b}

\[\begin{aligned}
    x_1[n] &= 2 x[n] + x_2[n] \\ 
    x_2[n] &= -0.5 x_5[n] + x_3[n] \\ 
    x_3[n] &= 2 x_5[n-1] \\ 
    x_4[n] &= x[n] + x[n-1] \\ 
    x_5[n] &= y[n] = x_4[n] + 0.5 x_3[n] \\
    &= x[n] + 2 x[n-1] + 0.5 x_5[n-1] + 2 x_5[n-2] \\ 
\end{aligned}\]

\subhomep{c}

即:

\[Y(z) (1 - 0.5 z^{-1} - 2 z^{-2}) = X(z) (1 + 2 z^{-1})\]

\[H(z) = \frac{1+2z^{-1}}{1 - 0.5 z^{-1} - 2z^{-2}}\] 

极点为 \(0.25 \pm j 0.25 \sqrt{33}\) ,在单位圆外,非 BIBO 稳定。

\subhomep{d}

\[\begin{aligned}
    n = 0, & y[0] = 1 \\ 
    n = 1, & y[1] - 0.5 = 2.5, y[1] = 3 \\ 
    n = 2, & y[2] -3.5 = 1.25, y[2] = 4.75
\end{aligned}\]


\homep{7.2}

\subhomep{a}

容限如 \figref{06} 

\qfig[06]{1606.png}{}

边界条件:

\[\begin{aligned}
    (\frac{\Omega}{\Omega_c T_d})^{2N} &= \frac{1}{|H_c(j\Omega)|^2} - 1 \\ 
    (\frac{0.2 \pi}{\Omega_c T_d})^{2N} &= \frac{1}{0.89125^2} - 1 \\ 
    (\frac{0.3\pi}{\Omega_c T_d})^{2N} &= \frac{1}{0.177832^2} - 1 \\ 
\end{aligned}\]

\subhomep{b}

解得 \(N = \lceil 5.8857 \rceil = 6\) , \(\Omega_c T_d = 0.704744\) 

\subhomep{c}

由于 \(s = \sigma + j \omega\):

\[H(s) = \frac{1}{1 + (s / (j \Omega_c T_d)^{2N})}\] 

可以看到这个式子的形式和前问一致,因此得到结果也一致。

\homep{7.4}

\subhomep{a}

极点为 \(e^{-0.2}\) 和 \(e^{-0.4}\) ,此时 \(2/T_d = 1\) 

\[H_c(s) = \frac{1}{s{+0.1}} - \frac{0.5}{s+0.2}\]

不唯一,复频域的周期性:

\[H_c(s) = \frac{1}{s{+0.1+2\pi j}} - \frac{0.5}{s+0.2+2\pi j}\]


\subhomep{b}

双线性变化是唯一的: \(s = \frac{1-z^{-1}}{1+z^{-1}} \frac{2}{T_d}\) 

得到 \(z = \frac{1+s}{1-s}\) 

那么:

\[H_c(s) =  \frac{2}{1-e^{-0.2}\frac{1-s}{1+s}} -  \frac{1}{1-e^{-0.4}\frac{1-s}{1+s}}\]



\homep{7.22}

\subhomep{a}

\(s = \frac{1-z^{-1}}{1+z^{-1}} \frac{2}{T_d}\) 

带入得到:

\[H(z) = \frac{T_d}{2} \frac{1 + z^{-1}}{1 - z^{-1}}\]

\[h[n] = \frac{T_d}{2} (u[n] + u[n-1]) \]

\subhomep{b}

根据上一问的系统函数:

\[\frac{T_d}{2} (x[n] + x[n-1]) = y[n] - y[n-1]\]

这个系统不稳定:单位圆存在极点

\subhomep{c}

\[H(e^{j\omega}) = \frac{T_d}{2} \frac{1 + e^{-j\omega}}{1 - e^{-j\omega}}\]

\[H(j\Omega) = \frac{1}{j \Omega}\]

对比如 \figref{07} ,但是模拟响应延伸到无穷远,在低频是良好的近似

\qfig[07]{1607.png}{}

\subhomep{d}

\[G(z) = \frac{2}{T_d} \frac{1-z^{-1}}{1+z^{-1}}\]

\[g[n] = \frac{2}{T_d} ((-1)^n u[n] - (-1)^{n-1} u[n-1])\]

\subhomep{e}

\[
\begin{aligned}
    G(e^{j\omega}) &= \frac{2}{T_d} \frac{1-e^{-j\omega}}{1+e^{j\omega}} \\ 
    &= \frac{2}{T_d} j \tan  \frac{\omega}{2}
\end{aligned}    
\]


而 \(G(j\Omega) = j \Omega\) ,在低频近似较好。

\subhomep{f}

在采样频率一致是可逆。

\homep{7.5} 

由经验设计法: \(\delta_2 = 40 dB\) , \(\Delta \omega = 0.05 \pi \) 

\[M = \frac{\delta_2 - 7.95}{2.286 \Delta \omega} = 89.2549\]

\[\beta = 0.5849 (19)^{0.4} + 0.07886 (19) = 3.39532\]

那么 \(N = 91\) , \(\beta = 3.39532\) 。

这是线性相位系统,对称中心为 \(n = 45\) ,那么

\homep{7.15}

在通带,\(\delta_p = -20 \log 0.05 = 26 dB\) ,在阻带, \(\delta_s = -20 \log 0.01 = 20 dB\) ,那么设计衰减至少为 \(26 dB\) ,因此根据表 7.1 ,只能选择: Hanning , Hamming , Blackman。此外,需要考虑过渡带宽度 \(0.1\pi\) 。

对于前两种窗函数: \(8 \pi / M = 0.1 \pi, M = 80\) ,最后一种: \(12 \pi / M = 0.1 \pi, M = 120 \pi\) 。

最小长度分别为 \(81\) , \(81\) , \(121\) 。


\homep{7.16}

\(\Delta \omega = 0.02 \pi\) 

\(\delta_2 = -20 \log \min (0.1, 0.05) = 33.979\) :

\(M = \frac{\delta_2 - 7.95}{2.286 \Delta \omega} = 181.221\)

同时 \(\beta = 2.65229\)

\homep{7.33}

\subhomep{a}

如\figref{08} ,群延时为 \(k\) 

\qfig[08]{1608.png}{}

\homep{b}

由于 \(\omega = 0\) 存在相位突变,因此需要在 \(z = 1\) 的零点,那么这是一个奇对称的序列,即 III 或 IV 类滤波器。

\subhomep{c}

\[\begin{aligned}
    h_d[n] &= \frac{1}{2\pi} \int_{-\pi}^\pi H(e^{j\omega}) e^{j\omega n } \dd{\omega} \\ 
    &=  \frac{1}{2\pi} \int_{-\pi}^0 e^{j(0.5\pi - \omega k)} e^{j\omega n } \dd{\omega} \\ 
    &+  \frac{1}{2\pi} \int_{0}^\pi e^{j(-0.5\pi - \omega k)} e^{j\omega n } \dd{\omega} \\
    &= \frac{-1}{2\pi(n-k)} (2 - \cos (\pi(n - k )))  \\ 
    &= \frac{1-\cos (\pi (n-k)}{\pi (n - k)}
\end{aligned}\]

\subhomep{d}

\(M = 21\) 时, \(N = 22\) ,对称中心为 \(10.5\) ,那么延时为 \(10.5\) 样本。

此时其系统零点为 \(1\) ,如\figref{09} 

\qfig[09]{1609.png}{}

\subhomep{e}

\(M = 20\) 时, \(N = 21\) ,对称中心为 \(10\) ,那么延时为 \(10\) 样本。

此时其系统零点为 \(\pm 1\) ,如\figref{10} 

\qfig[10]{1610.png}{}

% Start Here


% End Here

\end{document}