\documentclass[lang=cn,11pt,a4paper,cite=authoryear]{elegantpaper}

% 微分号
\newcommand{\dd}[1]{\mathrm{d}#1}
\newcommand{\pp}[1]{\partial{}#1}

% FT LT ZT
\newcommand{\dtft}[1]{\text{DTFT}[#1]}
\newcommand{\dtftr}[1]{\text{DTFT}^{-1}[#1]}
\newcommand{\dtfta}{\xrightarrow{\text{DTFT}}}

\newcommand{\where}[1]{\Big|_{#1}}
\newcommand{\abs}[1]{\left| #1 \right|}
\newcommand{\zt}[1]{\mathscr{Z}[#1]}
\newcommand{\ztr}[1]{\mathscr{Z}^{-1}[#1]}
\newcommand{\zta}{\xrightarrow{\mathscr{Z}}} 
\newcommand{\lt}[1]{\mathscr{L}[#1]}
\newcommand{\lta}{\xrightarrow{\mathscr{L}}} 
\newcommand{\ft}[1]{\mathscr{F}[#1]}
\newcommand{\fta}{\xrightarrow{\mathscr{F}}} 
\newcommand{\dsum}{\displaystyle\sum}
\newcommand{\aint}{\int_{-\infty}^{+\infty} }

% 积分求和号

% 简易图片插入
\newcommand{\qfig}[3][nolabel]{
  \begin{figure}[!htb]
      \centering
      \includegraphics[width=0.6\textwidth]{#2}
      \caption{#3}
      \label{#1}
  \end{figure}
}

\usepackage{pdfpages}
\includepdfset{pagecommand=\thispagestyle{plain}}
% 表格
\renewcommand\arraystretch{1.5}

\usepackage{shapepar}
\usepackage{longtable}
% 日期


\title{数字信号处理\quad 第一周作业}
\author{范云潜 18373486}
\institute{微电子学院 184111 班}
\date{\zhtoday}

\begin{document}

\maketitle

作业内容: 2.1, 2.2, 2.3; 2.12, 2.21, 2.22, 2.55;

% \tableofcontents
\setlength{\columnseprule}{0.4pt}
% Start Here
\begin{multicols*}{2}


    \homep{2.1}

    a) \(\trans{x(n)} = g(n) x(n)\) 

    \begin{itemize}
        \item 若是 \(g(n)\) 有界,稳定;其他,不稳定
        \item 因果
        \item 时变
        \item 无记忆
    \end{itemize}

    b) \(\trans{x(n)} = \sum_{k=n_0}^n x(k)\) 

    \begin{itemize}
        \item 不稳定
        \item 非因果
        \item 线性
        \item 时变
        \item 有记忆
    \end{itemize}

    c) \(\trans{x(n)} = \sum_{k=n-n_0}^{n+n_0} x(k)\) 

    \begin{itemize}
        \item 稳定
        \item 因果
        \item 线性
        \item 时不变
        \item 有记忆
    \end{itemize}

    d) \(\trans{x(n)} = x(n-n_0)\)

    \begin{itemize}
        \item 稳定
        \item 因果
        \item 线性
        \item 时不变
        \item 有记忆
    \end{itemize}

    e) \(\trans{x(n)} = e^{x(n)}\)

    \begin{itemize}
        \item 稳定
        \item 因果
        \item 非线性
        \item 时不变
        \item 无记忆
    \end{itemize}

    f) \(\trans{x(n)} = ax(n) + b\) 

    \begin{itemize}
        \item 稳定
        \item 因果
        \item 非线性
        \item 时不变
        \item 无记忆
    \end{itemize}

    g) \(\trans{x(n)} = x(-n)\)

    \begin{itemize}
        \item 稳定
        \item 非因果
        \item 线性
        \item 时变
        \item 有记忆
    \end{itemize}

    h) \(\trans{x(n)} = x(n) + 3 u(n+1)\)

    \begin{itemize}
        \item 稳定
        \item 因果
        \item 非线性
        \item 时变
        \item 无记忆
    \end{itemize}

\homep{2.2}

a)  \[\begin{aligned}
    y(n) &= h(n) \otimes x(n) \\ 
    &= \sum_{k=-\infty}^{\infty} h(k) x(n-k) \\
    &= \sum_{k=N_0}^{N_1} h(k) x(n-k)
\end{aligned}\]

那么必然有 

\[N_2 \leq n - k \leq N_3\] 

得到

\[N_0 + N_2 \leq n \leq N_1 + N3\]

b) 

由 a) 得到取值的范围,最终长度为 \(M+N-1\)

\homep{2.3}

\[\begin{aligned}
    y(n) &= u(n) \otimes h(n) \\
    &= \sum_{k = -\infty}^{\infty} u(k) a^{-n+k} u(k-n)\\
    &= \sum_{k=0}^n a^{-n+k} \\ 
    &= \frac{a^{-n} - a}{1 - a}
\end{aligned}\]

\homep{2.12}

a) 由于松弛特性

\[y(n) = 0, n < 0\]

之后使用递推

\[
\begin{aligned}
    y(0) &= 1\\ 
    y(1) &= 1 + 0 = 1\\
    y(2) &= 1 \cdot 2 + 0 = 2\\
    y(3) &= 2 \cdot 3 + 0 = 6\\
    \cdots\\
    y(n) &= n! 
\end{aligned}    
\]

归纳可得到 

\[y(n) = u(n) n!\]

b) 对这样的零状态系统,只需考虑输入,对于 \(a \delta(n)\)  引起的输出改变,可以推导得到

\[y(n) = a \cdot u(n) n!\]

故而,这是线性的

c) 对于 \(\delta(n-1)\) 重复推导得到,

\[y(n) = u(n-1) n!\]

因此是时变的

\homep{2.21}

设存在一个 \(n_0\) 使得 \(\trans{x(n_0)} \neq 0\) ,那么

\[\begin{aligned}
    \trans{x(n_0) + x(n_1)} &= y(n_0)+ y(n_1) \neq 0 \\
    \trans{x(n_1) + x(n_1)} &= y(n_1)+ y(n_1) = 0 
\end{aligned}\]

而这两式相等,因此矛盾,得证

\homep{2.22}

a) 

\[\{\darr 0 ,1\} \otimes \{\darr 2, 1\} = \{\darr 0,2,1\}\]

b)

\[\{\darr 2, -1\} \otimes \{\darr -1, 2, 1\} = \{\darr -2, 5, 0, -1\}\]

c)

\[
    \begin{aligned}
        &\{\darr 1,1,1,1,1\} \otimes \\
        & \{\darr  0,0,1,1,1,1,1,1,0,0,0,1,1,1,1,1,1\} = \\
        & \{\darr 0,0,1,2,3,4,5,5,4,3,\\
        & 2,2,2,3,4,5,5,4,3,2,1\}
    \end{aligned}
\]

d)



\[
\begin{aligned}
    &\{-2, 2, \darr 1, 1\} \otimes \{\darr 1, -1, 0, 0, 1, 1\} = \\
    & \{  -2,4  ,\darr -1,0  -3,0,3,2,1\}
\end{aligned}    
\]

\homep{2.55}

会。

随着 \(\omega\) 的变化,采样点的值会周期性变化,又因为第一个系统无记忆,那么 \(w(n)\) 同样有周期性,之后两个系统同理。

\end{multicols*}

% End Here

\end{document}