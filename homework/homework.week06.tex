\documentclass[lang=cn,11pt,a4paper,cite=authoryear,twocolumn]{elegantpaper}

% 微分号
\newcommand{\dd}[1]{\mathrm{d}#1}
\newcommand{\pp}[1]{\partial{}#1}

% FT LT ZT
\newcommand{\dtft}[1]{\text{DTFT}[#1]}
\newcommand{\dtftr}[1]{\text{DTFT}^{-1}[#1]}
\newcommand{\dtfta}{\xrightarrow{\text{DTFT}}}

\newcommand{\where}[1]{\Big|_{#1}}
\newcommand{\abs}[1]{\left| #1 \right|}
\newcommand{\zt}[1]{\mathscr{Z}[#1]}
\newcommand{\ztr}[1]{\mathscr{Z}^{-1}[#1]}
\newcommand{\zta}{\xrightarrow{\mathscr{Z}}} 
\newcommand{\lt}[1]{\mathscr{L}[#1]}
\newcommand{\lta}{\xrightarrow{\mathscr{L}}} 
\newcommand{\ft}[1]{\mathscr{F}[#1]}
\newcommand{\fta}{\xrightarrow{\mathscr{F}}} 
\newcommand{\dsum}{\displaystyle\sum}
\newcommand{\aint}{\int_{-\infty}^{+\infty} }

% 积分求和号

% 简易图片插入
\newcommand{\qfig}[3][nolabel]{
  \begin{figure}[!htb]
      \centering
      \includegraphics[width=0.6\textwidth]{#2}
      \caption{#3}
      \label{#1}
  \end{figure}
}

\usepackage{pdfpages}
\includepdfset{pagecommand=\thispagestyle{plain}}
% 表格
\renewcommand\arraystretch{1.5}

\usepackage{shapepar}
\usepackage{longtable}
% 日期


\title{数字信号处理\quad 第六周作业}
\author{范云潜 18373486}
\institute{微电子学院 184111 班}
\date{\zhtoday}

\begin{document}

\maketitle

作业内容:2.32,
2.33,
2.42,
3.40,
3.41,
5.1,
5.4, 
5.12;

\homep{2.32}

\[\begin{aligned}
    x[n] &= \cos \frac{\pi}{2} n = \frac{1}{2}(e^{j \frac{n \pi}{2}} + e^{-j \frac{n \pi}{2}}) \\
    H(e^{j\frac{\pi}{2}}) &= e^{-j\frac{\pi}{4}} \frac{1+e^{-j\pi}+4e^{-2j\pi}}{1+\frac{1}{2}e^{-j\pi}} = e^{-j\frac{\pi}{4}} 8 \\ 
    H(e^{-j\frac{\pi}{2}}) &= e^{j\frac{3\pi}{4}} \frac{1+e^{j\pi}+4e^{2j\pi}}{1+\frac{1}{2}e^{j\pi}} = - e^{-j\frac{\pi}{4}} 8 \\ 
    y[n] &= 4 (e^{j\frac{\pi}{2} n - j \frac{\pi}{4}} - e^{-j \frac{\pi}{2}n - j \frac{\pi}{4}}) 
\end{aligned}\]

\homep{2.33}

\[\begin{aligned}
    x[n] &= \cos (\frac{3\pi n}{2} + \frac{\pi}{4}) \\
    & = \frac{1}{2}(e^{j(\frac{3\pi}{2}n + \frac{\pi}{4})} + e^{-j(\frac{3 \pi n}{2} + \frac{\pi}{4})})
\end{aligned}\]

幅度响应恒为 \(1\) ,仅考虑相位响应:

\[\arg H(e^{j \frac{3\pi}{2}}) = \frac{2 \pi}{3} = - \arg H(e^{-j \frac{3\pi}{2}})\] 

\[\begin{aligned}
    y[n] &= \frac{1}{2}(e^{j(\frac{3\pi}{2}n + \frac{11}{12}n)} + e^{-j(\frac{3 \pi n }{2} + \frac{11 \pi }{12})})\\
     &=\cos (\frac{3\pi n }{2} + \frac{11 \pi}{12})
\end{aligned}\]

\homep{2.42}

\subhomep{a}


\[\begin{aligned}
    y[n] &= x[n] \otimes (h_2[n] + h_1[n] \otimes h_2[n])\\
    h[n] &= h_2[n] + h_1[n] \otimes h_2[n] \\ 
    &= \alpha ^n u[n] + \beta \alpha^{n-1} u[n-1]
\end{aligned}\] 

\subhomep{b}

转换到变换域:

\[\begin{aligned}
    H(z) &= H_1(z) H_2(z) +H_2(z)\\ 
    H_1(z) &= \beta \frac{1}{z} \\ 
    H_2(z) &= \frac{z}{z-\alpha} \\
    \therefore \: H(z) &= \frac{z + \beta}{z - \alpha}
\end{aligned}\]

\subhomep{c}

展开得到:

\[(1-\alpha / z)Y(z) = (1 + \beta / z) X(z)\]

进行逆变换

\[y[n] - \alpha y[n-1] = x[n] + \beta x[n-1]\]

\subhomep{d} 

因果;在 \(|\alpha| < 1\) 时稳定,此时极点在单位圆内。

\homep{3.40}

\subhomep{a}

变换域:  \[H(z)(X(z)-W(z)) + E(z) = W(z)\]

\[W(z) = X(z) \frac{H(z)}{1+H(z)} + E(z) \frac{1}{1+H(z)}\]

\subhomep{b}

\[H_1(z) = \frac{1/(z-1)}{1 + 1/(z-1)} = \frac{1}{z}\]

\[H_2(z) = \frac{1}{1 + 1 / (z-1)} = \frac{z-1}{z}\]

\subhomep{c}

由于是因果系统,\(z\) 向外扩展。\(H(z)\) 在单位圆上存在极点,因此不稳定;\(H_1(z), H_2(z)\) 极点在单位圆内部,因此稳定。

\homep{3.41} 

\subhomep{a} 稳定, \(r_{min} < 1 < r_{max}\)

\subhomep{b} 
\(v[n] = a^{-n} x[n], w[n] = y[n] a^{-n}\)

变换域:

\(V(z) = X(az), W(z) = Y(az)\)

\(\therefore\: G(az) = H(z), G(z) = H(z/a)\)

\(\therefore\: g[n] = a^n h[n]\)

\subhomep{c}

收敛域带入: \(0<r_{min}< |z/a| < r_{max} < \infty\) , 即: \(0 < |a r_{min}| < |z| < |a r_{max}| < \infty\) 。

\homep{5.1}

\[Y(e^{j \omega}) = \sum_{n=0}^{10} e^{-j\omega n } = \frac{\sin (11\omega/2)}{\omega/2} e^{-5\omega}\]

对于 \(X(e^{j\omega})\) 的每一频率取值,
都有 \(X(e^{j\omega}) H(e^{j\omega}) = Y(e^{j\omega})\) ,由于 \(\omega\) 在此区间连续,因此需要满足 \(y[n] = x[n]\) 且截止频率覆盖整个频段,\(\omega_c = \pi\) 。

\homep{5.4}

\subhomep{a}

\(\begin{aligned}
    X(z) &= \frac{z}{z-1/2}-\frac{z}{z-2}, 1/2 < |z| < 2\\
    Y(z) &= 6 \frac{z}{z-1/2} , |z| > \frac{3}{4}\\
    H(z) &= \frac{Y(z)}{X(z)} \\
    &= \frac{z-2}{z-3/4}, |z| > \frac{3}{4}
\end{aligned}\)

零极点如 \figref{01}

\qfig[01]{hw06p1.png}{零极点图}

\subhomep{b}

\(H(z) = \frac{z}{z-3/4} - \frac{2}{z}\frac{z}{z-3/4}\)

\(h[n] = (\frac{3}{4})^n u[n] - 2 (\frac{3}{4})^{n-1} u[n-1]\)

\subhomep{c}

\[\frac{Y(z)}{X(z)} = \frac{1-2/z}{1-3/(4z)}\]

\[y[n] - \frac{3}{4}y[n-1] = x[n] - 2 x[n-1]\]

\homep{5.12}

\subhomep{a}

极点:\(z=\pm 0.9 j\) 在单位圆内,因此稳定。

\subhomep{b}

单位圆外的因子:\(z=\pm 3\),那么

\[H_{ap} = \frac{1-9z^{-2}}{1-z^{-2}/9}\]

\[H_1(z) = \frac{1+0.2z^{-1}}{1+0.8z^{-2}} (1-\frac{1}{9}z^{-2})\]z



% Start Here

% End Here

\end{document}