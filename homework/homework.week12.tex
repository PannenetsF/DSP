\documentclass[lang=cn,11pt,a4paper,cite=authoryear]{elegantpaper}

% 微分号
\newcommand{\dd}[1]{\mathrm{d}#1}
\newcommand{\pp}[1]{\partial{}#1}

% FT LT ZT
\newcommand{\dtft}[1]{\text{DTFT}[#1]}
\newcommand{\dtftr}[1]{\text{DTFT}^{-1}[#1]}
\newcommand{\dtfta}{\xrightarrow{\text{DTFT}}}

\newcommand{\where}[1]{\Big|_{#1}}
\newcommand{\abs}[1]{\left| #1 \right|}
\newcommand{\zt}[1]{\mathscr{Z}[#1]}
\newcommand{\ztr}[1]{\mathscr{Z}^{-1}[#1]}
\newcommand{\zta}{\xrightarrow{\mathscr{Z}}} 
\newcommand{\lt}[1]{\mathscr{L}[#1]}
\newcommand{\lta}{\xrightarrow{\mathscr{L}}} 
\newcommand{\ft}[1]{\mathscr{F}[#1]}
\newcommand{\fta}{\xrightarrow{\mathscr{F}}} 
\newcommand{\dsum}{\displaystyle\sum}
\newcommand{\aint}{\int_{-\infty}^{+\infty} }

% 积分求和号

% 简易图片插入
\newcommand{\qfig}[3][nolabel]{
  \begin{figure}[!htb]
      \centering
      \includegraphics[width=0.6\textwidth]{#2}
      \caption{#3}
      \label{#1}
  \end{figure}
}

\usepackage{pdfpages}
\includepdfset{pagecommand=\thispagestyle{plain}}
% 表格
\renewcommand\arraystretch{1.5}

\usepackage{shapepar}
\usepackage{longtable}
% 日期


\title{数字信号处理\quad 第一周作业}
\author{范云潜 18373486}
\institute{微电子学院 184111 班}
\date{\zhtoday}

\begin{document}

\maketitle

作业内容:8.1, 8.2, 8.6, 8.30 

% \tableofcontents

\homep{8.1} 

\subhomep{a}

\[\begin{aligned}
    x[n] = x_c(n T) &= \sum_{k = -9}^{9} a_k \exp (j \frac{2\pi k n}{6}) 
\end{aligned}\]

各个分量周期的最小公倍数为\(N = 6\) 。

\subhomep{b}

\(\Omega_c = 18 \pi \times 10^3 \text{ rad } /s\) ,恰好不混叠发生在 \(\omega_c = \pi\) ,解得 \(T_{\min} = \dfrac{1}{18} \times 10^{-3}s\) ,那么已经混叠。

\subhomep{c}

\[\begin{aligned}
    \tilde{x}[n] &= \text{ IDFS } [\tilde{X}(k)] = \frac{1}{N} \sum_{k = 0}^{N-1} \tilde{X}(k) W_N^{-n k} \\ 
    &= \sum_{0}^{5} \tilde{x}[n] \exp(-j \pi n k / 3) \\
    &= \sum_{n = 0}^5 \sum_{m=-9}^{9} a_m \exp (j n m \pi / 3)  \\ 
    \\ &\quad\quad\quad\quad\quad \exp (-j \pi n k / 3) \\ 
    &= \sum_{m = -9} ^{9} a_m \dfrac{1 - \exp \dfrac{j (m - k) 2 \pi N}{6}}{1 - \exp \dfrac{j (m - k) 2 \pi}{6}} \\  
    &= 6 \sum_{m = -9}^{9} \sum_{M=-\infty}^{\infty} a_{k + 6 M}
\end{aligned}\]

\homep{8.2}

\subhomep{a}

DFS 代表在单位圆对 \(Z\) 变化的采样。

\[X(z) = \sum_{0}^{N-1} \tilde{x}[n] z^{-n} \] 

\[X_3(z) = \sum_{0} ^{3 N - 1} \tilde{x}[n] z^{n} = X(z) ( 1  +z^{-N} + z^{-2N})\]  

\[\begin{aligned}
    X_3(e^{-j 2\pi k / (3 N)}) &= X(e^{-j 2\pi k / (3N)}) \\
    & (1 + e^{-j 2\pi k / (3N)} + e^{-j 4\pi k / (3N)}) \\ 
    &= \tilde{X}(k/3)  (1 + e^{-j 2\pi k / (3N)} \\
    & + e^{-j 4\pi k / (3N)}) \\ 
    &= \left\{\begin{aligned}
        3 \tilde{X}(k/3), & k = 3m \\ 
        0, & \text{ else}
    \end{aligned}\right. 
\end{aligned}\]

\subhomep{b}
 
\(N=2\) 时, \(\tilde{X}(k) = W_2^0 + 2W_2^k = 1 + 2 \exp (-j \pi k)\) ;

\(N=6\) 时, \(\tilde{X}(k) = W_6^0 + 2W_6^k + W_6^{2k} + 2W_6^{3k} + W_6^{4 k} + 2 W_6^{5k}= (1 + 2 \exp (-j \pi k) )(1 + \exp(-j\pi k /3) + \exp(-j 2 \pi k / 3))\) 。满足。


\homep{8.6}

\homep{a}

\[x[n] \rightarrow \sum_{n=0}^{N-1} e^{j(\omega_0 - \omega) n} = \frac{e^{j(\omega_0 - \omega)N}}{1 - e^{j(\omega_0 - \omega)}}\]  

\homep{b}

\[\begin{aligned}
    x[k] &\rightarrow \sum_{n=0}^{N-1} e^{j(\omega_0 - 2 \pi k / N) n}\\
    &= \frac{e^{j(\omega_0 - 2\pi k / N)N}}{1 - e^{j(\omega_0 - 2\pi k / N)}} 
\end{aligned}\]  

\subhomep{c}

\[\begin{aligned}
    x[k] &\rightarrow  \frac{1}{1 - e^{j(2\pi k_0 / N - 2\pi k / N)}} 
\end{aligned}\]

\homep{8.30} 

\subhomep{a}

\[x[n] = \frac{1}{64} W_{64}^{-n 32} = \frac{e^{-j n\pi}}{64}\]

唯一,因为全部采样,没有丢失信息。

\subhomep{b}

按照上一问的方式,得到 \(x_n = \frac{e^{j \pi n 3 / 3}}{192} = \frac{e^{j\pi n}}{192}\) ,若是考虑到重叠,那么 \(x_n =  \frac{e^{-j n\pi}}{64} , \text{ if } 0 \leq n \leq 63; 0, \text{ else} \)  。

% Start Here

% End Here

\end{document}