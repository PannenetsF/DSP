\documentclass[lang=cn,11pt,a4paper,cite=authoryear, twocolumn]{elegantpaper}

% 微分号
\newcommand{\dd}[1]{\mathrm{d}#1}
\newcommand{\pp}[1]{\partial{}#1}

\newcommand{\homep}[1]{{\Large\textbf{Problem #1}}}
\newcommand{\subhomep}[1]{{\large\textbf{SubProblem #1}}}
\usepackage{mathrsfs} 
% FT LT ZT
\newcommand{\ft}[1]{\mathscr{F}[#1]}
\newcommand{\fta}{\xrightarrow{\mathscr{F}}}
\newcommand{\lt}[1]{\mathscr{L}[#1]}
\newcommand{\lta}{\xrightarrow{\mathscr{L}}}
\newcommand{\zt}[1]{\mathscr{Z}[#1]}
\newcommand{\zta}{\xrightarrow{\mathscr{Z}}}
\newcommand{\ztra}{\xrightarrow{\mathscr{Z}^{-1}}}
\newcommand{\dtft}[1]{\text{DTFT}[#1]}
\newcommand{\dtftr}[1]{\text{DTFT}^{-1}[#1]}
\newcommand{\dtfta}{\xrightarrow{\text{DTFT}}}
\newcommand{\where}[2]{\left.#1\right|_{#2}}

\newcommand{\trans}[1]{{T}[#1]}

% 积分求和号

\newcommand{\dsum}{\displaystyle\sum}
\newcommand{\aint}{\int_{-\infty}^{+\infty}}

% 简易图片插入
\newcommand{\qfig}[3][nolabel]{
  \begin{figure}[!htb]
      \centering
      \includegraphics[width=0.4\textwidth]{#2}
      \caption{#3}
      \label{#1}
  \end{figure}
}

% 表格
\renewcommand\arraystretch{1.5}

\usepackage{multicol}

% 日期

\newcommand{\darr}{\underset{\mathop{\uparrow}}}

\title{数字信号处理\quad 第十五周作业}
\author{范云潜 18373486}
\institute{微电子学院 184111 班}
\date{\zhtoday}

\begin{document}

\maketitle

作业内容:9.19,9.21,9.28,9.48;10.1;10.4,10.5,10.9

\homep{9.19}

在 \(8\) 点 FFT 中, \(X(\exp (j 6 \pi / 8))\) 对应的是 \(k = 3\) 。

根据 Goertzel 算法及其流程图:

\[\left\{
\begin{aligned}
    2 \cos \frac{2\pi \cdot 3}{8} & = a \\ 
    -W_8^3 &= b 
\end{aligned}    
\right.\]

解得 \(a = -\sqrt{2}, b = \dfrac{1 + j}{\sqrt{2}}\) 。

\homep{9.21} 

\subhomep{a}

\[H(z) = \sum_{n = 0} ^\infty \left(W_N^k z^{-1} \right)^n = \frac{1}{1-W_N^k z^{-1}}   \]

那么:

\[\begin{aligned}
    y_k[n] &= \sum_{m=0}^n x[m] W_N^{k(n-m)} \\ 
    y_k[N] &= \sum_{m=0}^n x[m] W_N^{k(N-m)} \\ 
    &= \sum_{m=0}^n x[m] W_N^{-km} \\ 
\end{aligned}\]

同时:

\[\begin{aligned}
    X[k] &= \sum_{n=0}^{N-1} x[n] W_N^{kn} \\ 
    X[N-k] &= \sum_{n=0}^{N-1} x[n] W_N^{(N-k)n} \\ 
    &= \sum_{n=0}^{N-1} x[n] W_N^{-kn} \\ 
\end{aligned}\]

显然 \(X[N-k] = y_k[N]\) 成立。

\subhomep{b} 

将书中 (9.9 式) 改写为:

\[\begin{aligned}
    H_k(z) &= \frac{1 - W_N^{-k} z^{-1}}{(1 - W_N^{-k} z^{-1}) (1 - W_{N}^{k}z^{-1})} \\ 
    &= \frac{1}{1 - W_N^{-k} z^{-1}} 
\end{aligned}\]

显然,同 (a) 中的系统函数一致,因此两者功能一致。

\homep{9.28}

\subhomep{a}

有效频率间隔为 \(\Delta f = \frac{1}{NT} = \frac{10000}{N} \leq 50\) ,解得 \(N \geq 200\) 。因此最小取 \(256\) 。

\subhomep{b}

对要求进行分析:

\[\begin{aligned}
    Y(z) &= X(0.8 z) \\ 
    Y(1.25 z) &= X(z) \\ 
    y[n] (1.25 z)^{-n} &= x[n] z^{-n} \\
    y[n] &= 1.25^n x[n]
\end{aligned}\] 

\homep{9.48}

步骤如下所示:

\begin{enumerate}
    \item 对两边取 \(512\) 点 FFT ,得到向量形式如: 
    \[Y = A \cdot Y + B \cdot X \rightarrow Y = \frac{B}{1-A} X \]  
    \item 那么 \(H(z)\) 的 FFT 为 \(\dfrac{B}{1-A}\) ,取其序列中 \(k = 2\) 的值!
\end{enumerate}

\homep{10.1} 

\subhomep{a}


\[\begin{aligned}
    \omega &= \frac{150}{10000}\cdot 2\pi = 0.03 \pi \\ 
    \Omega &= \frac{\omega}{T} = 300 \pi \text{ rad} / s
\end{aligned}\]

\subhomep{b}


\[\begin{aligned}
    \omega &= \frac{800}{10000}\cdot 2\pi = 0.16 \pi \\ 
    \Omega &= \frac{\omega}{T} = 1600 \pi \text{ rad} / s
\end{aligned}\]

\homep{10.4}

\subhomep{a}

根据实序列变换的对称性: \(X[k] = X^*[-k]\) ,又根据 DFT 的周期性:

\[X[200] = X^*[200] = X^*[800]1-j\] 

\subhomep{b}

根据采样关系: 

\[X(j \Omega) = T X[k], \text{ where } \Omega = \frac{\omega}{T} = \frac{w\pi k }{N T} \]  

同时,由于此时只存在 \(\omega \in [-\pi, \pi]\) ,将 \(k = 800\) 转换到 \(k = 200\) 。

\[\begin{aligned}
    \Omega_1 &= \frac{2\pi 200}{1000} 20000 = 8000 \pi \text{ rad}/s , X = \frac{1-j}{20000}\\ 
    \Omega_2 &= \frac{2\pi -200}{1000} 20000 = -8000 \pi \text{ rad}/s, \\
    & \quad X = \frac{1+j}{20000}\\  
\end{aligned}\]

\homep{10.5}

\(\cos \Omega_0 t\) 对应 \(\omega = \dfrac{2\pi k_0}{T N} = \Omega_0\) ,那么 \(T = \dfrac{2\pi k_0}{N\Omega_0}\) 。此外 \(T = \dfrac{2\pi k_0}{N(2\pi - \Omega_0)}\) 同样满足。

\homep{10.9}

\[\begin{aligned}
    \Delta \Omega &= \frac{\pi}{64} \\
    \Delta \Omega &= \frac{5 \pi}{64} \\
    \Delta \Omega &= \frac{5 \pi}{64} 
\end{aligned}\]

可以看出第一个间隔太小,虽然第二三个间隔一致,但是第三个次分量幅度极小,难以检测,因此第二个可以分出。


% \tableofcontents

% Start Here

% End Here

\end{document}