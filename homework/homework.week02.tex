\documentclass[lang=cn,11pt,a4paper,cite=authoryear]{elegantpaper}

% 微分号
\newcommand{\dd}[1]{\mathrm{d}#1}
\newcommand{\pp}[1]{\partial{}#1}

\newcommand{\homep}[1]{{\Large\textbf{Problem #1}}}
\newcommand{\subhomep}[1]{{\large\textbf{SubProblem #1}}}
\usepackage{mathrsfs} 
% FT LT ZT
\newcommand{\ft}[1]{\mathscr{F}[#1]}
\newcommand{\fta}{\xrightarrow{\mathscr{F}}}
\newcommand{\lt}[1]{\mathscr{L}[#1]}
\newcommand{\lta}{\xrightarrow{\mathscr{L}}}
\newcommand{\zt}[1]{\mathscr{Z}[#1]}
\newcommand{\zta}{\xrightarrow{\mathscr{Z}}}
\newcommand{\ztra}{\xrightarrow{\mathscr{Z}^{-1}}}
\newcommand{\dtft}[1]{\text{DTFT}[#1]}
\newcommand{\dtftr}[1]{\text{DTFT}^{-1}[#1]}
\newcommand{\dtfta}{\xrightarrow{\text{DTFT}}}
\newcommand{\where}[2]{\left.#1\right|_{#2}}

\newcommand{\trans}[1]{{T}[#1]}

% 积分求和号

\newcommand{\dsum}{\displaystyle\sum}
\newcommand{\aint}{\int_{-\infty}^{+\infty}}

% 简易图片插入
\newcommand{\qfig}[3][nolabel]{
  \begin{figure}[!htb]
      \centering
      \includegraphics[width=0.4\textwidth]{#2}
      \caption{#3}
      \label{#1}
  \end{figure}
}

% 表格
\renewcommand\arraystretch{1.5}

\usepackage{multicol}

% 日期

\newcommand{\darr}{\underset{\mathop{\uparrow}}}

\title{数字信号处理\quad 第二周作业}
\author{范云潜 18373486}
\institute{微电子学院 184111 班}
\date{\zhtoday}

\begin{document}

\maketitle

作业内容:作业:
2.8,
2.11,
2.17,
2.44;
3.1(b)、(g),
3.3,
3.4,
3.6(c),
3.9,
3.28;


% \tableofcontents

\homep{2.8}

\[h(n) = 5 (-\frac{1}{2})^n u(n), x(n) = (\frac{1}{3})^n u(n)\]

对应 

\[H(z) = 5 \frac{z}{z + 1/2}, |z| > \frac{1}{2}\]

\[X(z) = \frac{z}{z - 1/3}, |z| > \frac{1}{3}\]

那么 

\[\frac{Y(z)}{z} = 5 \frac{z}{(z+1/2)(z-1/3)}\]

解得

\[Y(z) = 2 \frac{z}{z - 1/3} + 3 \frac{z}{z + 1/2}, |z| > \frac{1}{2}\]

\[y(n) = 3 (- \frac{1}{2})^n u(n) + 2 (\frac{1}{3})^n u(n)\]

\homep{2.11}

由于 LTI 不会改变信号的频率,分解

\[x(n) = \frac{1}{2 j} \left(e^{j \frac{\pi}{4} n} - e^{-j \frac{\pi}{4} n}\right)\]

那么 

\[y(n) = \frac{1}{2 j} \left(H(e^{j \pi/4})e^{j \pi n / 4  } - H(e^{-j \pi/4}) e^{-j \pi n / 4} \right)\]


带入公式

\[H(e^{j \pi / 4}) = \frac{1 - (- j)}{1 - 1/2} = 2\sqrt{2}e^{j \pi/4}\]     

\[H(e^{-j \pi / 4}) = \frac{1 - j}{1 - 1/2} = 2\sqrt{2}e^{-j \pi/4}\]     

代回

\[y(n) = 2 \sqrt{2} \sin (\frac{\pi}{4} (n+1))\]

\homep{2.17}

\subhomep{a}

\[r(n) = G_{M+1}(n)\]

那么 

\[R(e^{j \omega}) = \frac{\sin(1/2 \cdot \omega(M+1))}{\sin(1/2 \cdot \omega)} e^{-j \frac{M}{2} \omega}\]


\subhomep{b}

如\figref{217}

\qfig[217]{h201m20.png}{\(w(n)\) 在 \(M=20\) 的示意图}

\subhomep{c}

\[
\begin{aligned}
    W(e^{j\omega}) &= R(e^{j\omega}) \otimes (\sum_{n=-\infty}^{\infty} \frac{1}{2} (1 - \cos \frac{2 \pi n}{M}) e^{-j\omega n}) \\
    &= R(e^{j\omega}) \otimes (\sum_{n=-\infty}^{\infty} \frac{1}{2} (1 - \frac{e^{j 2 \pi n / M} + e^{-j 2 \pi n / M}}{4}) e^{-j\omega n}) \\
    &= R(e^{j \omega}) \otimes (\frac{1}{2} \delta(\omega) -\frac{1}{4} \delta(\omega + \frac{2 \pi }{M}) - \frac{1}{4} \delta(\omega - \frac{2 \pi}{M}))\\
    &= \frac{R(e^{j\omega})}{2} + (-\frac{1}{4})(R(e^{j(\omega + 2\pi/M)}) + R(e^{j(\omega - 2\pi/M)}))
\end{aligned}
\]

如\figref{2171}

\qfig[2171]{h202.png}{\(w(n)\) 在 \(M=20\) 的示意图}


\homep{2.44}

\subhomep{a} 

\[\left.\dtft{x(n)}\right|_{\omega=0} = \sum_n x(n)  = 6\]

\subhomep{b}

\[\where{\dtft{x(n)}}{\omega = \pi} = - \sum_n x(n) = -6\]

\subhomep{c} 

序列关于 \(n = 2\) 对称,得到实数

\[\dtft{x(n)} = \sum_{n} x(n) e^{-j\omega(n-2)} e^{-2 j \omega}\]

那么辐角为 \(-2\omega\)

\subhomep{d}

\[\int_{-\pi}^\pi X(e^{j \omega}) \dd{\omega} = 2 \pi \dtftr{X(e^{j\omega})}\where{}{n=0} = 2 \pi\]

\subhomep{e}

\[X(e^{-j\omega}) = \sum_n x(n) e^{j\omega n} = \sum_n e^{-j\omega n} = \dtft{x(-n)}\]

如\figref{d2171}

\qfig[d2171]{h203.png}{(e) 题解}

\subhomep{f}

\[\text{Re}(X(e^{j\omega})) = \frac{1}{2}(X(e^{j\omega}) + X^*{(e^{-j\omega})}) = \dtft{X_e(n)}\]

对于实序列

\[X_e(n) = \frac{1}{2}(x(n) + x(-n))\]

计算得到

\[\text{Re}(X(e^{j \omega})) = -\cos 7 \omega + \cos 5 \omega + 2 \cos 4 \omega + 2 \cos \omega + 2\]



如\figref{d21712}

\qfig[d21712]{h204.png}{(f) 题解}

\homep{3.1}

\subhomep{b}

\[\zt{-(\frac{1}{2})^nu(-n-1)} = -\frac{z}{z-1/2}, \text{ where } |z| < \frac{1}{2}\]

\subhomep{g}

\[\zt{(\frac{1}{2})^n G_{9}(n)} = \frac{1 - 1/(2z)^{10}}{1-1/2z}, \text{ where } |z| \neq 0\]


\homep{3.3}

\subhomep{a} \[X_a(z) = \frac{z}{z-a} + \frac{z}{z - 1/a}, \text{ where } a < |z| < \frac{1}{a}\]

\subhomep{b}

\[X_b(z) = \frac{1-1/z^N}{1-1/z} = \frac{z^N - 1}{z^{N-1}(z-1)}\]

\subhomep{c}

三角形状的函数一般是由门函数卷积而得,由于起点为 0 ,需要移位

\[x_c(n) = x_b(n) \otimes x_b(n-1)\]

\[X_c(z) = X_b(z) \cdot \frac{X_b(z)}{z} = \frac{1}{z^{2 N - 1}} (\frac{z^N - 1}{z-1})^2\]


如\figref{d3}

\qfig[d3]{h301.png}{收敛域}

\homep{3.6}

\subhomep{c 部分分式}

\[\frac{X(z)}{z} = \frac{z - 1/2}{z^2 + \frac{3}{4} z + \frac{1}{8}} = \frac{4}{z+1/2}-\frac{3}{z+1/4}, \text{ where } |z| > \frac{1}{2}\]

\[x(n) = 4 (-\frac{1}{2})^n u(n) - 3 (-\frac{1}{4})^n u(n)\]

\subhomep{c 展开法}

\[\frac{1-\frac{1}{2}z^{-1}}{1+\frac{3}{4}z^{-1}+\frac{1}{8}z^{-2}} = 1 - \frac{5}{4}z^{-1} + \frac{13}{16}z^{-2} - \frac{29}{64}z^{-3}\cdots\]

观察可得,存在较大较小的两项使得其在正负之间变化,并且均为 2 的幂次,最终得到

\[x(n) = 4 (-\frac{1}{2})^n u(n) - 3 (-\frac{1}{4})^n u(n)\]

\homep{3.9}

\subhomep{a}

\[H(z) = \frac{2 z}{z - 1/2} - \frac{z}{z+1/4}\]

又因为是因果系统,收敛域为 \(|z| > 1/2\)

\subhomep{b}

稳定,\(|z| = 1\) 在收敛域内

\subhomep{c}

\[y(n) \zta -\frac{1}{3} \frac{z}{z+1/4} + \frac{4}{3} \frac{z}{z-2}\]

那么

\[X(z) = \frac{Y(z)}{H(z)} = \frac{z-1/2}{z-2} = \frac{1}{4} + \frac{3}{4} \frac{z}{z-2}\]

所以

\[x(n) = \frac{1}{4} \delta(n) - \frac{3}{4} 2^nu(-n-1)\]

\homep{3.28}

\subhomep{a}

设 
\[x_0(n) \zta \frac{3}{(1-\frac{1}{4}z^{-1})^2} = X_0(z)\]

那么 
\[X_0(z) = \frac{3 z}{z - {1}/{4}} + \frac{3}{4}z \cdot \frac{1}{(z-1/4)^2}\]

设
\[\frac{\dd{X_1(z)}}{\dd{z}} = -\frac{1}{(z-1/4)^2}\]

那么
\[X_1(z) = \frac{1}{z-1/4} = 4(-1+\frac{z}{z-1/4})\]

那么
\[\frac{3}{4}z \cdot \frac{1}{(z-1/4)^2} = \frac{3}{4}(-z)\frac{\dd{X_1(z)}}{\dd{z}} \ztra \frac{3}{4}nx_1(n)\]


\[x_0(n) = 3(-1)\frac{1}{4^n}u(-n-1) + \frac{3}{4}n(-4\delta(n)+(-1)\frac{1}{4^n}u(-n-1)) = -3n\delta(n) + (-3-\frac{3}{4}n)\frac{n}{4}u(-n-1)\]

那么
\[x(n) = x_0(n-3) = -3(n-3)\delta(n-3)+(-3-\frac{3}{4}(n-3))\frac{n-3}{4}u(-n+2)\]

\subhomep{b}

\[X(z) = \sum_{k=0}^{\infty}\frac{(-1)^k}{(2k+1)!}z^{2k+1}\]

\[x(n) = \sum_{k=0}^{\infty}\frac{(-1)^k}{(2k+1)!}\delta(n+2k+1)\]


\subhomep{c}


\[X(z) = z^7 + \frac{-1}{1-z^{-7}} = z^7 + (-1) \sum_{n=0}^{\infty}(z^{-7})^n\]

\[x(n) = \delta(n+7)-\sum_{k=0}^{\infty}\delta(n-7k)\]
% Start Here
% End Here

\end{document}