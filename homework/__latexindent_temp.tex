\documentclass[lang=cn,11pt,a4paper,cite=authoryear,twocolumn]{elegantpaper}

% 微分号
\newcommand{\dd}[1]{\mathrm{d}#1}
\newcommand{\pp}[1]{\partial{}#1}

% FT LT ZT
\newcommand{\dtft}[1]{\text{DTFT}[#1]}
\newcommand{\dtftr}[1]{\text{DTFT}^{-1}[#1]}
\newcommand{\dtfta}{\xrightarrow{\text{DTFT}}}

\newcommand{\where}[1]{\Big|_{#1}}
\newcommand{\abs}[1]{\left| #1 \right|}
\newcommand{\zt}[1]{\mathscr{Z}[#1]}
\newcommand{\ztr}[1]{\mathscr{Z}^{-1}[#1]}
\newcommand{\zta}{\xrightarrow{\mathscr{Z}}} 
\newcommand{\lt}[1]{\mathscr{L}[#1]}
\newcommand{\lta}{\xrightarrow{\mathscr{L}}} 
\newcommand{\ft}[1]{\mathscr{F}[#1]}
\newcommand{\fta}{\xrightarrow{\mathscr{F}}} 
\newcommand{\dsum}{\displaystyle\sum}
\newcommand{\aint}{\int_{-\infty}^{+\infty} }

% 积分求和号

% 简易图片插入
\newcommand{\qfig}[3][nolabel]{
  \begin{figure}[!htb]
      \centering
      \includegraphics[width=0.6\textwidth]{#2}
      \caption{#3}
      \label{#1}
  \end{figure}
}

\usepackage{pdfpages}
\includepdfset{pagecommand=\thispagestyle{plain}}
% 表格
\renewcommand\arraystretch{1.5}

\usepackage{shapepar}
\usepackage{longtable}
% 日期


\title{数字信号处理\quad 第十四周作业}
\author{范云潜 18373486}
\institute{微电子学院 184111 班}
\date{\zhtoday}

\begin{document}

\maketitle

作业内容:9.6,9.7,9.14,9.17,9.26,9.27

% \tableofcontents

% Start Here

\homep{9.6}

无法判断, DIF 和 DIT 都有含有 \(W_N^2\) 的层,分别在倒数第二层和第二层。

\homep{9.7}

易知, \(2 \pi k / N = 2 \pi 7 / 32\) 那么 \(k = 7\) ,对应的 \(\omega_k = 7\pi / 16\) 。

\homep{9.14}

对 DIT 来说,需要先计算子序列进而推出原始序列;而 DIF 是先分割子序列,设置合适的频点后逐个计算子序列。

对于 DIT :

\[
\begin{aligned}
    m = 1 &\rightarrow r = 0\\
    m = 2 &\rightarrow r = 0/4\\
    m = 3 &\rightarrow r = 0/2/4/6\\
    m = 4 &\rightarrow r = 0/1/2/3/4/5/6/7
\end{aligned}
\]

\homep{9.17}

根据上一小题中的总结,应该为 DIT 。

\homep{9.26}

由于是输入的扩展,需要应用 DIF:

\[\begin{aligned}
    Y[k] &= \sum_{n = 0}^{2N-1} y[n] W_{2N}^{k n}\\
    &= \sum_{n=0}^{N-1}y[n] W_{2N}^{k n} + (-1)^k \sum_{n=0}^{N-1}y[n] W_{2N}^{k n} \\
    &= (1+(-1)^k)\sum_{n=0}^{N-1}y[n] W_{2N}^{k n} \\
    &= \left\{
        \begin{aligned}
            & 2 X[k/2] , k = 0, 2, 4 \cdots\\
            & 0, k = 1, 3, 5 \cdots \\
        \end{aligned}    
    \right.
\end{aligned}\]

% End Here

\end{document}