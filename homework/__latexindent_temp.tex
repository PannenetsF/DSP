\documentclass[lang=cn,11pt,a4paper,cite=authoryear,twocolumn]{elegantpaper}

% 微分号
\newcommand{\dd}[1]{\mathrm{d}#1}
\newcommand{\pp}[1]{\partial{}#1}

\newcommand{\homep}[1]{{\Large\textbf{Problem #1}}}
\newcommand{\subhomep}[1]{{\large\textbf{SubProblem #1}}}
\usepackage{mathrsfs} 
% FT LT ZT
\newcommand{\ft}[1]{\mathscr{F}[#1]}
\newcommand{\fta}{\xrightarrow{\mathscr{F}}}
\newcommand{\lt}[1]{\mathscr{L}[#1]}
\newcommand{\lta}{\xrightarrow{\mathscr{L}}}
\newcommand{\zt}[1]{\mathscr{Z}[#1]}
\newcommand{\zta}{\xrightarrow{\mathscr{Z}}}
\newcommand{\ztra}{\xrightarrow{\mathscr{Z}^{-1}}}
\newcommand{\dtft}[1]{\text{DTFT}[#1]}
\newcommand{\dtftr}[1]{\text{DTFT}^{-1}[#1]}
\newcommand{\dtfta}{\xrightarrow{\text{DTFT}}}
\newcommand{\where}[2]{\left.#1\right|_{#2}}

\newcommand{\trans}[1]{{T}[#1]}

% 积分求和号

\newcommand{\dsum}{\displaystyle\sum}
\newcommand{\aint}{\int_{-\infty}^{+\infty}}

% 简易图片插入
\newcommand{\qfig}[3][nolabel]{
  \begin{figure}[!htb]
      \centering
      \includegraphics[width=0.4\textwidth]{#2}
      \caption{#3}
      \label{#1}
  \end{figure}
}

% 表格
\renewcommand\arraystretch{1.5}

\usepackage{multicol}

% 日期

\newcommand{\darr}{\underset{\mathop{\uparrow}}}

\title{数字信号处理\quad 第八周作业}
\author{范云潜 18373486}
\institute{微电子学院 184111 班}
\date{\zhtoday}

\begin{document}

\maketitle

作业内容:4.2,4.3,4.4,4.5


\homep{4.2}

此处 \(\omega = 0.25 \pi \) ,那么 \(\Omega = \omega / T = 250 \pi\) ;类似的 \(\omega = 0.25 \pi + 2 \pi \) 时, \(\Omega = 2250 \pi \) 。

\homep{4.3} 

\subhomep{a} 

\[\omega = \frac{\pi}{3} = 4000\pi \frac{1}{T} , T = 12000\]

\subhomep{b} 

\[\omega = \frac{\pi}{3} + 2 \pi = 4000\pi \frac{1}{T} , T = \frac{12000}{7}\]

\homep{4.4} 

\subhomep{a} 

\[\left\{\begin{aligned}
    \omega_1 &= \frac{\pi}{5} = 20 \pi / T \\
    \omega_2 &= \frac{2 \pi }{5} = 40 \pi / T
\end{aligned}\right., \therefore\: T = 100\]

\subhomep{b} 

\[\left\{\begin{aligned}
    \omega_1 &= \frac{\pi}{5} + 2k_1 \pi = 20 \pi / T \\
    \omega_2 &= \frac{2 \pi }{5} + 2 k_2 \pi = 40 \pi / T
\end{aligned}\right., \therefore\: T = \frac{100}{10k_1 + 1} = \frac{100}{10k_2 + 2}\]

而 \(k_i\) 为整数,因此不存在其他解。

\homep{4.5}

\subhomep{a}

\[\Omega_S = 2\pi / T = 2.5 k \cdot 2 \pi, T = 10^{-5} s\]

\subhomep{b}

\[\omega = \Omega T,\: \Omega_S = \frac{\omega}{T}, \: f_s = \Omega_S/2\pi = 625Hz\]

\subhomep{c}

同上,\[f_S = 2 f_{S,(b)} = 1250 Hz\]




% \tableofcontents

% Start Here

% End Here

\end{document}