\documentclass[lang=cn,11pt,a4paper,cite=authoryear,twocolumn]{elegantpaper}

% 微分号
\newcommand{\dd}[1]{\mathrm{d}#1}
\newcommand{\pp}[1]{\partial{}#1}

% FT LT ZT
\newcommand{\dtft}[1]{\text{DTFT}[#1]}
\newcommand{\dtftr}[1]{\text{DTFT}^{-1}[#1]}
\newcommand{\dtfta}{\xrightarrow{\text{DTFT}}}

\newcommand{\where}[1]{\Big|_{#1}}
\newcommand{\abs}[1]{\left| #1 \right|}
\newcommand{\zt}[1]{\mathscr{Z}[#1]}
\newcommand{\ztr}[1]{\mathscr{Z}^{-1}[#1]}
\newcommand{\zta}{\xrightarrow{\mathscr{Z}}} 
\newcommand{\lt}[1]{\mathscr{L}[#1]}
\newcommand{\lta}{\xrightarrow{\mathscr{L}}} 
\newcommand{\ft}[1]{\mathscr{F}[#1]}
\newcommand{\fta}{\xrightarrow{\mathscr{F}}} 
\newcommand{\dsum}{\displaystyle\sum}
\newcommand{\aint}{\int_{-\infty}^{+\infty} }

% 积分求和号

% 简易图片插入
\newcommand{\qfig}[3][nolabel]{
  \begin{figure}[!htb]
      \centering
      \includegraphics[width=0.6\textwidth]{#2}
      \caption{#3}
      \label{#1}
  \end{figure}
}

\usepackage{pdfpages}
\includepdfset{pagecommand=\thispagestyle{plain}}
% 表格
\renewcommand\arraystretch{1.5}

\usepackage{shapepar}
\usepackage{longtable}
% 日期


\title{数字信号处理\quad 第九周作业}
\author{范云潜 18373486}
\institute{微电子学院 184111 班}
\date{\zhtoday}

\begin{document}

\maketitle

作业内容:4.8,4.20,4.15,4.18

\homep{4.8}

\subhomep{a} 

显然其频带受限, 为了不混叠,\(\Omega_N \leq \pi/T\) ,因此 \(T_{max} = 0.5 \times 10^{-4} s\) 。

\subhomep{b}

\[\begin{aligned}
    y[n] &= x[n] \otimes h[n] \\
    &= \sum_{k=-\infty}^\infty x[k] h[n-k] \\
    \therefore\: h[n] &= T u[n]
\end{aligned}\]

\subhomep{c} 

\[\begin{aligned}
    y[n] &= T \sum_{k = -\infty} ^{n} x[k] e^{-j\omega k} |_{\omega = 0} \\ 
    &= T X(e^{j\omega})
\end{aligned}\]

\subhomep{d} 

\[\int_{-\infty} ^\infty x_c(t) e^{-j \Omega t} \dd{t} = X_c{e^{j\Omega}} |_{\Omega = 0}\] 

又因为

\[X(e^{j\omega}) = \frac{1}{T} \sum_{-\infty} ^\infty X_c(j \Omega + k \Omega_s) |_{\Omega = \omega / T}\] 

为了采样不失真,那么采样应满足采样定理

\[T < 2 T_s = 10^{-4} s \] 

\homep{4.20} 

\subhomep{a}

在频谱搬移中,不会发生混叠,则 \(\omega_0 = \Omega_0 T \leq \pi \) , \(T_{max} = \pi / \Omega_0\) , \(F_s = 1/T_S = 2000 Hz\) 。

\subhomep{b}

在滤波时,为了不损失信号 \(\omega_0 \leq \pi / 2\) ,同上, \(F_S = 4000 Hz\) 。

\homep{4.15} 

显然,只能恢复 \(\dfrac{\pi}{3}\) 内的信号。

\subhomep{a}

\(|\omega|_{max} = \pi/4\) ,\(x[n] = x_r[n]\) 。

\subhomep{b}

\(|\omega|_{max} = \pi/2\) ,\(x[n] \neq x_r[n]\) 。

\subhomep{c}

时域的乘法看作是频域的窗函数卷积,那么

\(|\omega|_{max} = \pi/8 \times 2 = \pi/4\) ,\(x[n] = x_r[n]\) 。

\homep{4.18} 

先进行升采样再进行滤波, 不失真需要满足 \(\omega_0 / L < \min{\pi/M, \pi/L}\) 。分别带入得到:

a) \(\omega_0 < dfrac{2}{3} \pi\) , b) \(\omega_0 < \dfrac{3}{5} \pi\) , c) \(\omega_0 < \pi \)

% \tableofcontents

% Start Here

% End Here

\end{document}