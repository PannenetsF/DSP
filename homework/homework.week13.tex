\documentclass[lang=cn,11pt,a4paper,cite=authoryear]{elegantpaper}

% 微分号
\newcommand{\dd}[1]{\mathrm{d}#1}
\newcommand{\pp}[1]{\partial{}#1}

% FT LT ZT
\newcommand{\dtft}[1]{\text{DTFT}[#1]}
\newcommand{\dtftr}[1]{\text{DTFT}^{-1}[#1]}
\newcommand{\dtfta}{\xrightarrow{\text{DTFT}}}

\newcommand{\where}[1]{\Big|_{#1}}
\newcommand{\abs}[1]{\left| #1 \right|}
\newcommand{\zt}[1]{\mathscr{Z}[#1]}
\newcommand{\ztr}[1]{\mathscr{Z}^{-1}[#1]}
\newcommand{\zta}{\xrightarrow{\mathscr{Z}}} 
\newcommand{\lt}[1]{\mathscr{L}[#1]}
\newcommand{\lta}{\xrightarrow{\mathscr{L}}} 
\newcommand{\ft}[1]{\mathscr{F}[#1]}
\newcommand{\fta}{\xrightarrow{\mathscr{F}}} 
\newcommand{\dsum}{\displaystyle\sum}
\newcommand{\aint}{\int_{-\infty}^{+\infty} }

% 积分求和号

% 简易图片插入
\newcommand{\qfig}[3][nolabel]{
  \begin{figure}[!htb]
      \centering
      \includegraphics[width=0.6\textwidth]{#2}
      \caption{#3}
      \label{#1}
  \end{figure}
}

\usepackage{pdfpages}
\includepdfset{pagecommand=\thispagestyle{plain}}
% 表格
\renewcommand\arraystretch{1.5}

\usepackage{shapepar}
\usepackage{longtable}
% 日期


\title{数字信号处理\quad 第十三周作业}
\author{范云潜 18373486}
\institute{微电子学院 184111 班}
\date{\zhtoday}

\begin{document}

\maketitle

作业内容:8.9, 8.14, 8.15, 8.16, 8.23, 8.27

\homep{8.9} 

\subhomep{a} 

\[X(k) = \sum_{n=0}^{N-1} x[n] e^{-j\frac{2\pi}{N} k n }\]

\(N=5, k=2\) 即可。对于任一有理频点 \(\omega = \frac{P}{Q}\pi\) ,若是 \(P\) 为奇数, \(N = 2Q\) ,\(P\) 为偶数, \(N=Q\) 。

\subhomep{b}

\(L=27\)。

\homep{8.14}

\[x_3[2] = 3 \time 1 + 1 \times 2 + 2 \times 2 = 9\]

\homep{8.15}

卷积得到 \(\{1, 1 + 2 a, a, a + 2\}\) ,得到 \(a = -1\) 。

\homep{8.16}

以 \(4\) 为周期拓延,考虑交叠后 \(1 + b = 4\) , \(b = 3\) 。

\homep{8.23} 

\subhomep{a}

可以直接进行 DFT ,只需在不满的部分补 \(0\) 。

\subhomep{b}

存在交叠:

\[\begin{aligned}
    x'[n] &= \frac{1}{N} \sum_{k=0}^{N-1} X(k) W_N^{-k n} \\
    &= \frac{1}{N} \sum_{k=0}^{N-1} \left(\sum_{m=-\infty}^{+\infty}x[m] e^{j\frac{2\pi}{N} k m}\right) W_N^{-k n} \\ 
    &= \sum_{m=-\infty}^{+\infty} x[m] \sum_{k=-\infty}^{+\infty} \delta[k-n+rN] \\
    &= \sum_{m=-\infty}^{+\infty} x[m - r N]
\end{aligned}\]

将 \(x[n]\) 按照以上方法进行交叠,再进行 DFT 。

\homep{8.27}

\subhomep{a} 

\(y_{a} = \{1,2,\cdots 10,10,\cdots 10,9,\cdots,2,1\}\), 长度为 \(110\) 。如 \figref{01} 。

\qfig[01]{hw1401.png}{}

\subhomep{b} 

\(y_{b} = \{10, 10, \cdots 10\}\), 长度为 \(100\) 。
如 \figref{02} 。

\qfig[02]{hw1402.png}{}


\subhomep{c} 

\(y_{c} = y_{a}\), 长度为 \(110\) 。如 \figref{01} 。


% \tableofcontents

% Start Here

% End Here

\end{document}