\documentclass[en,11pt,english,black,simple,device=ppt]{elegantbook}

\def\mainclass{main}

\title{Some}
% \subtitle{数字设计初步}

\author{Pannenets.F}
% \institute{微电子学院}
\date{\today}
% \version{4}
% \bioinfo{分类}{笔记}

\extrainfo{Je reviendrai et je serai des millions. —— <<Spartacus>>}
\setcounter{tocdepth}{3}

\lstset{
  mathescape = false}
% \logo{logo-blue.png}
% \cover{logo.jpg}

% 微分号
\newcommand{\dd}[1]{\mathrm{d}#1}
\newcommand{\pp}[1]{\partial{}#1}

\newcommand{\homep}[1]{{\Large\textbf{Problem #1}}}
\newcommand{\subhomep}[1]{{\large\textbf{SubProblem #1}}}
\usepackage{mathrsfs} 
% FT LT ZT
\newcommand{\ft}[1]{\mathscr{F}[#1]}
\newcommand{\fta}{\xrightarrow{\mathscr{F}}}
\newcommand{\lt}[1]{\mathscr{L}[#1]}
\newcommand{\lta}{\xrightarrow{\mathscr{L}}}
\newcommand{\zt}[1]{\mathscr{Z}[#1]}
\newcommand{\zta}{\xrightarrow{\mathscr{Z}}}
\newcommand{\ztra}{\xrightarrow{\mathscr{Z}^{-1}}}
\newcommand{\dtft}[1]{\text{DTFT}[#1]}
\newcommand{\dtftr}[1]{\text{DTFT}^{-1}[#1]}
\newcommand{\dtfta}{\xrightarrow{\text{DTFT}}}
\newcommand{\where}[2]{\left.#1\right|_{#2}}

\newcommand{\trans}[1]{{T}[#1]}

% 积分求和号

\newcommand{\dsum}{\displaystyle\sum}
\newcommand{\aint}{\int_{-\infty}^{+\infty}}

% 简易图片插入
\newcommand{\qfig}[3][nolabel]{
  \begin{figure}[!htb]
      \centering
      \includegraphics[width=0.4\textwidth]{#2}
      \caption{#3}
      \label{#1}
  \end{figure}
}

% 表格
\renewcommand\arraystretch{1.5}

\usepackage{multicol}

% 日期

\newcommand{\darr}{\underset{\mathop{\uparrow}}}

\begin{document}

\maketitle
\frontmatter


\mainmatter

\chapter{绪论}

\includepdf[pages=-]{chap/chap00.pdf}

\chapter{离散时间信号与系统}

\includepdf[pages=-]{chap/chap01-1.pdf}

\chapter{离散时间系统变换域分析}

\section{变换}

\includepdf[pages=-]{chap/chap02.pdf}

\section{频响}

\includepdf[pages=-]{chap/chap02-3.pdf}

\section{幅相特性}

\includepdf[pages=-]{chap/chap02-4.pdf}

\chapter{信号采样和重构}

\section{理想采样与重构}

\includepdf[pages=-]{chap/chap03-1-2.pdf}

\section{抽取和内插}

\includepdf[pages=-]{chap/chap03-3.pdf}

\section{量化噪声}

\includepdf[pages=-]{chap/chap03-4.pdf}

\chapter{离散傅里叶变换及快速算法}

\section{DFS}

\includepdf[pages=-]{chap/chap04-1-2.pdf}

\section{DFS}

\includepdf[pages=-]{chap/chap04-3.pdf}

\section{FFT}

\includepdf[pages=-]{chap/chap04-4.pdf}


\section{DFT 工程应用:估计连续傅里叶变换}

\includepdf[pages=-]{chap/chap04-5.pdf}

















\end{document}
